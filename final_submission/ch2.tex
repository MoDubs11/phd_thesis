\chapter[Warming can alter host behaviour to the same extent as behaviour-manipulating parasites]{Warming can alter host behaviour to the same extent as behaviour-manipulating parasites}
\label{chap:physbeh}

\section{Summary}
Parasites are ubiquitous and act as key regulators of the dynamics and stability of ecosystems by modifying the physiology and behaviour of their host organisms. It is, however, as yet unclear how parasitic relationships will act to moderate or accelerate the ecological impacts of global climate. We seek to discover how warming may moderate the effects of parasites on the physiology and behaviour of their hosts and, if so, how. Utilizing a well-established aquatic host-parasite model system --- the ecologically important amphipod \emph{Gammarus duebeni} and its Acanthocephalan parasite, \emph{Polymorphus minutus} --- I constructed full energy budgets for infected and uninfected hosts across a wide range of temperatures. I also analysed the phototactic and geotactic behaviour of the hosts, both infected and uninfected, across the same temperature range. Here I show that parasites and warming both independently alter host movement in the water column and host response to light. Moreover, a few degrees of warming has the capacity to alter these critical behaviours to the same extent as infection with behaviour-manipulating parasites. By enhancing host behavioural manipulation, warming may amplify the effects of parasites on ecological dynamics, particularly given the ubiquity of parasites in nature. 


\section{Introduction}

As climate change accelerates \citep{ipcc2014}, the importance of warming and its effects on ecosystems around the globe will continue to grow. While the direct effects of warming have been quantified for many species \citep{wernberg2012}, relatively little is known about its indirect effects, mediated by interspecific interactions \citep{shaver2000, post2008}. Though the nature and magnitude of these effects are considerably more difficult to predict \citep{sanford1999, wernberg2012}, they can be at least as important as direct effects in determining the overall impacts of warming on ecosystems \citep{chapin1983, post2008, donohue2017, kordas2017}. However, the complexity of ecological networks can lead to significant uncertainty in predictions, where even small changes in the abundance of a single species can determine the responses of entire communities to disturbance \citep{saterberg2013, mrowicki2016}. Ecological network complexity is amplified even further by the many different kinds of interactions that occur between species \citep{kefi2012}. Behaviour-mediated species interactions, in particular, have been shown to influence food webs \citep{schmitz1997}. Such interactions can, however, be crucial to understanding and predicting the effects of perturbations on ecosystems \citep{mcclean2015, suraci2016, donohue2017}. It has been shown that the temperature dependence of species interactions can contribute to overall resistance and resilience of ecosystems, particularly in terms of the disturbances related to climate change \citep{kordas2011, dell2014}. 

Parasites are ubiquitous \citep{torchin2004} and contribute significantly to diversity, biomass and energy flow in many ecosystems \citep{ lafferty2006parasites,  kruis2008, dunne2013, grabner2017}. Parasites increase connectance and link density within food-webs, and they can dominate trophic interactions \citep{lafferty2006parasites, amundsen2009}. Their biomass has been shown to exceed that of top predators and many other key consumer groups in some systems \citep{kruis2008, preston2013}. Therefore, though they remain remarkably underappreciated, parasites have the potential to play important roles in moderating the stability of ecosystems throughout the globe \citep{dunne2013}. 

In addition to their direct impacts on host physiology, parasites have long been known to impact the behaviour, metabolism, and survival of their host organisms \citep{burnett1949, kaldonski2009, perrot2012, poulin2013, perrot2014}. Many parasites alter the behaviour of ecologically important organisms to make them more susceptible to predation by suppressing predator-avoidance behaviours and thus facilitating completion of the parasite’s life-cycle \citep{dick2010, hatcher2014, toscano2014}. Such behavioural modifications have been demonstrated across a range of parasite species, with significant indirect consequences for non-parasitised individuals \citep{kadoya2015, demandt2018}, species interactions \citep{reisinger2015}, and overall ecosystem function \citep{sato2012}. However, we know little about the implications of warming for the strength, or indeed the nature, of parasitic manipulation of host behaviour. Recent experimental work suggests that parasitism and elevated temperature do not interact in moderating feeding rates of their hosts \citep{labaude2016}, but may modify specific light-avoidance behaviour \citep{labaude2017}. Links have yet to be drawn between the combined physiological and behavioural impacts of warming and parasitism. 

Many prey species have developed behavioural strategies to avoid capture. In aquatic ecosystems, two of the most common strategies employed, both individually and together, are negative phototaxis and geotaxis \citep{marriott1989, bauer2005}.  Phototaxis refers to an animal’s movement in response to light while geotaxis is the response of an animal to gravity; in rivers and lakes the response to gravity correlates with the position of the animal in the water body, either towards the surface or the bottom. We currently have little insight into the effects of warming and parasite infection on the geotactic and phototactic responses of host organisms across the current seasonal range of temperatures experienced in aquatic environments \citep{kaldonski2007, jacquin2014}. It is also crucial to consider the increased temperatures these ecosystems will face in future warming scenarios, as such knowledge is needed to predict how parasites will act to moderate th e functioning and stability of ecosystems in a warming world.

Here, I explore whether, and how, temperature moderates both physiological and behavioural components of host-parasite relationships, using a well-established host-parasite model system. I first explored whether temperature modifies the effects of parasites on host physiology by quantifying full energy budgets for individuals of the amphipod \emph{G. duebeni} in the presence and absence of infection with larvae of the acanthocephalan parasite \emph{P. minutus}. I then examined the individual and combined effects of temperature and parasitic infection on two commonly-exhibited behaviours in gammarid species that acanthocephalan parasites are known to manipulate: phototaxis and geotaxis \citep{bauer2005, perrot2012}. Both behaviours are important in this species for predator avoidance as they bring the amphipods further from their predators physically and encourage innate hiding behaviours \citep{bethel1973, cezilly2000, bauer2005, jacquin2014, perrot2016}. 

The relationship between \emph{P. minutus} and its intermediate host, \emph{G. duebeni}, is a particularly well-suited model system for this study, as parasitised individuals can be identified \emph{in situ} and the parasite is a known manipulator of host behaviour \citep{moore1996, fielding2003, benesh2005, perrot2007, jacquin2014}. \emph{P. minutus} infection is common in freshwater systems across Europe \citep{medoc2006} and Ireland (Fig. \ref{fig:acanth_map}. \emph{Gammarus} spp. are important detritus processers in freshwater ecosystems \citep{macneil1997, sutcliffe2000} and comprise a major component of the diets of many fish and water fowl around the globe \citep{mortensen1982, byers2010}.

The aims of our study were to (1) examine whether temperature moderates the effects of parasites on the physiology and/or behaviour of their hosts and, if so, (2) quantify the relative importance and nature of those effects. Metabolic theory suggests that temperature will increase the rates of many components of ecologically-relevant physiology \citep{brown2004, kordas2011}. The theory has also been shown to explain within-host parasite interactions as well as parasite impacts on host physiology \citep{kirk2018}. I predict that temperature will impact physiology, with higher rates of respiration at higher temperatures \citep{halcrow1967, bulnheim1979}, and higher feeding rates with increasing temperature \citep{nilsson1974}. As parasites have been shown to depress respiration rates in normoxic conditions \citep{perrot2016}, I expect antagonistic influences of temperature and parasites on this component of physiology. Additionally, I predict that temperature will not directly impact behaviour, but parasitic infection will encourage movement upwards and towards light \citep{labaude2017}. The multiple behaviours and physiological traits examined here under a wide range of temperature conditions will allow us to examine the interaction of temperature and parasitic infection in a novel way utilizing an ecologically important model species. 

\section{Method}

\subsection{Physiological assays}

I calculated energy budgets (following \citet{nilsson1974}) for both infected and uninfected \emph{G. duebeni} individuals at 3\degree C intervals between 3 and 18\degree C, as 3\degree C is the minimum temperature \emph{G. duebeni} experience in Ireland and survival rates of \emph{G. duebeni} fell below 25\% at temperatures above 18\degree C. \emph{G. duebeni} were collected via kick sample from Lough Ennell, Co. Westmeath, Ireland (53\degree 28'13.2"N 7\degree 23'00.2"W). Fifteen different \emph{Gammarus} individuals were used for each infection and temperature treatment combination (i.e.\ 15 individuals $\times$ 2 treatments $\times$ 6 temperatures = 180 individuals in total) and were acclimated from 15\degree C to their experimental temperatures with 3\degree C changes per day \citep{penk2016}. Following acclimation, \emph{G. duebeni} individuals were starved for 24 hours to ensure a clear digestive tract before the commencement of feeding trials. Individual \emph{Gammarus} without eggs and with a fresh mass of 10-20 mg were used in the experiment, as this size range encompassed most of the naturally infected \emph{G. duebeni} individuals in their source population.

For feeding trials, alder (\emph{Alnus glutinosa}) leaves were cut into 2.5 cm discs that were dried at 37\degree C for 12 hours, weighed and then conditioned in water from Lough Ennell for three days to enhance palatability. Leaf material was presented as the sole food source in accordance with standard methods \citep{agatz2014}. \emph{G. duebeni} individuals were housed in individual 60 ml glass jars containing 50 ml filtered aerated lake water and a clear glass pebble to allow the gammarids to express natural hiding behaviour. Individuals were presented with a leaf disc and allowed to feed for five days, over which time they were checked twice daily for molting and mortality. Five jars lacking \emph{G. duebeni} were used as procedural controls to quantify leaf leaching, or mass loss to the water and microbial breakdown \citep{webster1986}, at each temperature. After five days, each leaf disc was removed, dried at 37\degree C for 12 hours and weighed to four decimal places. After a subsequent 24-hour starvation, \emph{G. duebeni} were removed from the jars. To collect faecal matter, the water was filtered through pre-weighed Whatman glass-fibre GF/F filters, which were dried for 12 hours at 37\degree C. Filters and faeces were then ashed at 500\degree C for 4 hours and weighed to four decimal places to determine the relative amount of organic matter on the filter.  Leaf consumption rates (\emph{C}) were calculated as:

\begin{equation}
C=  \frac{(M_0-M_t)\times(1-L_T)}{I_t}
\end{equation}

where $M_0$ and $M_t$ are the mass of dried leaves at the commencement and end of the experiment, respectively, $L_T$ is the leaching correction, which corresponds to the mean proportional weight loss from leaching and handling of the leaves in procedural controls at the specific temperature ($T$) being tested \cite{nilsson1974}, and $I_t$ is the duration of the experiment (in days). \emph{C} was converted to calorific values using published values of 22.03 J mg$^{-1}$  alder leaf \citep{adcock1982}. For all physiological analyses, dry weight of the host was used to standardize measurements. 

Respiration rates were determined for each \emph{G. duebeni} individual used in the feeding experiments in sterilized, aerated lake water. Measurements were taken in a water bath in the dark to minimise stress using a YSI 53 biological oxygen monitoring system fitted with a DATAQ output tracker. Respiration chambers containing 5 ml of  $>$90\% oxygen-saturated sterile lake water were set up for each \emph{G. duebeni} individual with a magnetic stir bar separated from the amphipod by 350$\mu$m nylon mesh. The oxygen consumption of each \emph{G. duebeni} was calculated from the rate of change of oxygen concentrations in the closed systems over 35 minutes or until the concentration decreased by 20\% \citep{penk2016}. At the start and end of each set of respiration measurements, a sample of water was run without an amphipod present to act as a blank for the instrument. To account for instrument drift, blank samples were run every five samples throughout each set of measurements, which were consistently less than 5\% of measured rates. Respiration rates were standardized to blanks by subtracting the respiration rates of blanked samples and converted into energy equivalents using a value of 450 kJ mol$^{-1}$ O$_2$, as determined previously for freshwater crustaceans \citep{gnaiger1983}. 

Growth rate was calculated as the change in wet weight before and after experimental run divided by the number of days in the experiment, which was five. Assimilation and efficiency were calculated as per \citet{nilsson1974}. 

\subsection{Behavioural assays}
I quantified phototactic behaviour as per \citet{perrot2004} and geotactic behaviour as per \citet{bauer2005} in parasitised and unparasitised \emph{G. duebeni} individuals, at the same temperature intervals as in the physiological experiments. I measured both geotactic and phototactic behaviour for each individual at each temperature. The order of temperature exposures was randomized, as were the run orders of individuals through the experiments. All \emph{G. duebeni} were maintained in filtered, aerated lake water at 4\degree C with a 12h:12h light:dark cycle prior to all experimental assays. Individuals were acclimated to each experimental temperature with changes of 3\degree C per day in a constant temperature room. Overall, 390 \emph{G. duebeni}, comprising 195 infected and 195 uninfected individuals, were used in the assays, though final numbers varied across treatments due to mortality that occurred during the trials. Infected (mean $\pm$ S.E. fresh mass: 34.4 $\pm$ 1.2 mg) and uninfected (36.7 $\pm$ 1.3 mg) \emph{G. duebeni} individuals used in the assays had similar (t$_3_8_5$ = 1.39, \emph{p} = 0.16) body mass. For all assays, the water was changed between trials. 

Phototaxis assays were conducted by placing individual amphipods into one litre of aerated filtered lake water inside a clear plastic tank. Half of the tank was fully darkened on all sides using black plastic sheeting and the other half was exposed to a light source located 15 cm from the top of the tank, similar to methods described in \citet{perrot2004}. \emph{G. duebeni} individuals were given five minutes to acclimate to the set-up, after which the position of the individual, either in the light zone or the dark zone, was recorded every 30 seconds for five minutes. For each time point, a score of 0 was given if the individual was in the dark half and a score of 10 was given if the individual was in the well-lit half of the tank, leading to a minimum possible score of 0 (highly photophobic) and a maximum score of 100 (highly photophilic).  	

The geotactic behaviour of individuals was assessed using translucent graduated cylinders of 5 cm diameter. The inside of the cylinders was smooth, to prevent any interference from clinging behaviours. Geotaxis assays were always completed in darkness to eliminate potentially confounding responses to light. The cylinder was divided into ten zones of equal height; the bottom zone was scored 1 and the uppermost zone 10. Observations were conducted after individual \emph{G. duebeni} were introduced to the cylinder and allowed to acclimate to the experimental set up for five minutes. The position of individuals within the water column was then noted every 30 seconds for five minutes, leading to a geotaxis score between 10 (only interacted with the benthos) and 100 (only interacted with the surface).  

Upon completion of physiological and behavioural assays, all \emph{G. duebeni} individuals were weighed and dissected to verify absence of infection in the case of uninfected individuals or remove the larval \emph{P. minutus} from infected individuals. All individuals were checked for parasitic infections and only \emph{G. duebeni} naturally infected with single, cystacanth stage infections were included in analyses. \emph{G. duebeni} individuals were then dried overnight at 37\degree C and their dry weight was quantified. Larval parasites were placed in a solution of 0.25 mM sodium taurocholate at 37\degree C overnight to activate the cysts for identification purposes \citep{horvath1969} and their identity was confirmed with microscopic morphological examination using \citet{mcdonald1988}. 

\subsection{Data analyses}

Consumption rate, respiration rate, growth rate, assimilation and feeding efficiency (the ratio of assimilation to consumption expressed as a percentage) from the physiological assays were analysed as response variables with generalised additive models (GAMs) in mgcv:gam \citep{wood2016}.  Temperature (fixed, smoothed with thin-plate regression splines with four knots), infection status (fixed, two levels), and amphipod sex (fixed, two levels) were incorporated in all GAMs. To examine interactions between temperature and infection status, a smoothed term with 4 knots of temperature by infection status was incorporated in the GAMs. The GAMs generated were compared to generalised linear models (GLMs) with temperature, host sex, and infection status as fixed effects (Appendix Table A1). The GAMs are presented in this thesis, as they consistently had lower Akaike Information Criterion (AIC) values. 
Linear mixed-effects (LME) models, with temperature, infection status, and amphipod sex as fixed effects and \emph{G. duebeni} individual as a random effect, were used to analyse data from the behavioural assays, using lme4::lmer \citep{bates2015}. Likelihood ratios were used to determine \emph{p}-values by comparing the models with and without the tested effect or interaction term between temperature and parasitism present \citep{crawley2012}. All data were analysed with R version 3.4.1 \citep{r2017}.

\section{Results}

\subsection{Physiological assays}
Respiration rates increased with increasing temperature (Fig. \ref{fig:physio}a), but decreased with parasitic infection (Fig. \ref{fig:physio}a; Table \ref{tab:physiot}). There was a significant interaction between infection and temperature, where infection reduced respiration at lower temperatures but does not seem to impact respiration at higher temperatures as strongly (Fig. \ref{fig:physio}a; Table \ref{tab:physiot}). Temperature influenced feeding efficiency (Fig. \ref{fig:physio}b; Table \ref{tab:physiot}) with highest efficiencies in the middle of the temperature range. There was, however, no net effect of temperature on rates of consumption (Fig. \ref{fig:physio}c; Table \ref{tab:physiot}), growth (Fig. \ref{fig:physio}d; Table \ref{tab:physiot}), or assimilation (Fig. \ref{fig:physio}e; Table \ref{tab:physiot}). Sex of the adult amphipod impacted only growth rates (Fig. \ref{fig:physio}d; Table \ref{tab:physiot}). There was no main or interactive effect of parasitic infection on any of the physiological response variables I measured other than respiration rates. 

\begin{figure}[H]
    \centering
    \includegraphics[keepaspectratio,width=\textwidth]{figures/ch2/physio_1.pdf}
  \caption [Impact of temperature and parasites on host physiology]{Variation in rates (mean  $\pm$ S.E.) of (a) respiration, (b) feeding efficiency, (c) consumption, (d) growth, and (e) assimilation efficiency of \emph{G. duebeni} individuals infected (open circles) and uninfected (closed circles) with \emph{P. minutus}.} 
    \label{fig:physio}
\end{figure}

\begin{landscape}

{\small
\begin{table} [H]
\caption [Results of generalized additive models testing the effects of temperature, parasitic infection, and host sex on host physiology.]{Results of generalized additive models testing the effects of temperature, parasitic infection, and host sex on various components of host physiology. Significant terms are listed in bold.}
\centering
\begin{tabular}{ c  c  c  c  c  c  c  c  c  c  c  } \toprule
Response  & \multicolumn{2}{l}{Temperature} & \multicolumn{2}{l}{Infection} & \multicolumn{2}{l}{Host Sex} & \multicolumn{2}{l}{Temperature} & \multicolumn{2}{l}{Temperature} \\
                  &            &                    &               &               &                    &         & \multicolumn{2}{l}{$\times$Infection}     & \multicolumn{2}{l}{$\times$Infection}   \\ 
                  &            &                    &               &               &                    &         & \multicolumn{2}{l}{(Infected)}     & \multicolumn{2}{l}{(Uninfected)}   \\ \midrule
                  & F           & \emph{P}                & F              & \emph{P}             & F                  & \emph{P}        & F                & \emph{P}               & F           & \emph{P}                     \\
Consumption       & 3.48       & 0.09               & 1.04          & 0.31          & 0.8                & 0.37    & 1.84             & 0.27            & 0.08        & 0.82                 \\
Respiration       & \textbf{110.56}     &\textbf{$<$0.01}     & \textbf{0.51}          & \textbf{0.02}        & $<$0.01    & 0.97    & \textbf{8.63}             & \textbf{0.02}            & \textbf{12.32}       & \textbf{$<$0.01}    \\
Growth            & 2.22       & 0.24               & 0.35          & 0.56          & \textbf{4.17}               &\textbf{0.04}    & 0.39             & 0.61            & 0.13        & 0.77                 \\
Efficiency        & \textbf{9.52}       & \textbf{$<$0.01}     & 1.03          & 0.31          & 0                  & 0.99    & 0.92             & 0.44            & 0.27        & 0.67                 \\
Assimilation      & 1.72       & 0.15               & 1.16          & 0.29          & 0.03               & 0.87    & 0.01             & 0.94            & 3.98        & 0.11          \\ \bottomrule     
\end{tabular}
\label{tab:physiot}
\end{table}
}
\end{landscape}

\subsection{Behavioural assays}
Parasitism altered both the phototactic and geotactic behaviour of \emph{G. duebeni} (Fig. \ref{fig:taxis}; Table \ref{tab:taxistab}), with infected individuals displaying significantly greater preference for light and upward movement in the water column (Fig. \ref{fig:taxis}). Further, these effects were not modified by temperature (i.e.\ there were no interactions between temperature and parasitism; Fig. \ref{fig:taxis}). However, I found that temperature influenced geotaxis (Fig. \ref{fig:taxis}b; Table \ref{tab:taxistab}) in \emph{G. duebeni} independently, yet in a similar manner to parasitism; \emph{G. duebeni} individuals displayed significantly increased preference to move up into the water column as temperatures increased (Fig. \ref{fig:taxis}b). 

\begin{figure}[H]
    \centering
    \includegraphics[keepaspectratio,width=\textwidth]{figures/ch2/taxis_1.pdf}
  \caption [Impact of temperature and parasitic infection on predator-avoidance behaviour]{Variation in (mean $\pm$ S.E.) the (a) phototactic and (b) geotactic behaviour of \emph{G. duebeni} individuals infected (open circles, dotted lines) and uninfected (closed circles, solid lines) with \emph{P. minutus}. Relationships are shown as least-squares regression $\pm$ 95\% C.I.} 
    \label{fig:taxis}
\end{figure}

{\small 
\begin{table} [H]
\caption[Results of chi-squared analysis of linear mixed-effects models testing the impact of temperature, parasitic infection, and host sex on the geotactic and phototactic behaviour of \emph{G. duebeni}.]{Results of chi-squared analysis of linear mixed-effects models testing the impact of temperature, parasitic infection, and host sex on the geotactic and phototactic behaviour of \emph{G. duebeni}. Significant terms are listed in bold.}
\begin{tabular} { c  c  c  c  c  c  c  c  c } \toprule
Response & \multicolumn{2}{l}{Temperature} & \multicolumn{2}{l}{Infection} & \multicolumn{2}{l}{Host Sex} & \multicolumn{2}{l}{Temperature$\times$} \\ 
                 &        &             &        &                &            &        &             \multicolumn{2}{l}{Infection}            \\
                  & $\chi^2$         & P                  & $\chi^2$       & P                  & $\chi^2$                 & P       & $\chi^2$               & P               \\ \midrule
Phototaxis        & 0.97       & 0.32               & \textbf{5.19}     & \textbf{0.02}               & 0.55               & 0.46    & 0.09             & 0.77            \\
Geotaxis          &\textbf{92.33}      &\textbf{$<$0.01}     & \textbf{25.49}    & \textbf{$<$0.01}   & $<$0.01    & 0.95    & 0.77             & 0.38  \\    \bottomrule    
\end{tabular}
\label{tab:taxistab}
\end{table}
}


\section{Discussion}

My results demonstrate that temperature and parasites both significantly influence key components of predator-avoidance behaviour in gammarid amphipods. A key finding of my study was that just a few degrees of warming has the capacity to supress predator-avoidance behaviour to the same extent as infection with behavioural manipulating parasites, as a temperature increase of six degrees increased movement upward in the water column as much as parasitic infection. Warming encouraged movement upwards in the water column, reducing the use of benthic refugia. As temperature increases, ectothermic hosts become more capable of movement, likely allowing the movement upwards and toward light away from the benthos \citep{pawar2016, abram2017}. The altered movements not only increase the vulnerability of the arthropod to predators \citep{perrot2012} but also may simultaneously alter the time hosts spend foraging and predating, as they more frequently move away from benthic detritus and invertebrates towards surficial refugia or open water. The temperature-dependence of both trophic \citep{uszko2017} and non-trophic \citep{kordas2017} interactions between species comprises a key determinant of the response of whole communities to warming. My analysis of the temperature dependence of predator-avoidance behaviour reveals a previously underappreciated impact of warming that could be critical to predicting its impacts in many ecosystems

I found that warming and parasitism additively influenced the predator-avoidance behaviour of \emph{G. duebeni} and that these effects were consistent across a broad temperature range. Geotactic behaviour, associated with the use of benthic refugia, responded especially strongly and monotonically to warming. Such behavioural change, particularly in relatively common and functionally important species such as our focal host organism, has the capacity to modify the structure and dynamics of ecosytems, though future work is necessary to determine the degree to which our results impact in-situ communities \citep{kefi2015, kefi2016, suraci2016, donohue2017}. This novel and important finding was not recorded in previous studies of amphipod behaviour (e.g.\ \citet{labaude2017}), possibly because a more limited range of temperatures studied prevented the broad trend I found from being detected. By examining the full 15\degree C range experienced by \emph{G. duebeni}, I was able to see the equivalence of the impacts of infection and 6°C of warming. There is also specificity in host-parasite interactions, due to tightly coupled co-evolution, which might explain the differential observations between this study and prior work \citep{omahony2004, westram2011}. 

With recent findings on the complexity of \emph{P. minutus} lineages \citep{zittel2018}, it will be necessary to determine the generalisability of our findings by examining data from multiple strains of the parasite and multiple hosts. Future work to determine the commonality of these findings would shed light on the specificity of behaviour manipulation and potentially yield exciting findings for both ecology and parasitology. As relatively minor changes in temperature of only a few degrees – well within the range of seasonal temperature variability in northern temperate climates – altered geotaxis to the same extent as parasitic infection, the importance of examining the hosts’ entire realistic temperature range is clearly apparent. 

There is potential for the altered behaviour of the intermediate hosts under warmer conditions to increase the rate of completion of the parasite’s life cycle \citep{tierney1993, strepparava2017}.The specific impacts of temperature on the development of the parasite within the definitive host need to be examined, as responses of species to temperature can be asymmetric \citep{dell2014, goedknegt2015} and the rate of parasite development is crucial to understanding how temperature may influence the speed at which the parasites are transmitted. Given the significant role that parasites can play in moderating the dynamics of entire communities \citep{dunne2013}, these findings have implications for both the functioning and stability of ecosystems in a warming world, though a thorough assessment of the dynamics of the parasite in its definitive hosts would be necessary to confirm the prediction.

The effects of both parasitism and warming on predator-avoidance behaviour found in my study contrast with the comparatively minor effects of my focal parasites on host physiology in everything other than respiration rates. At the lowest temperatures, infection with the parasite suppressed respiration of the host, a finding in line with prior studies on acanthocephalan parasites \citep{rumpus1974}, though at higher temperatures there were no significant differences between infected and uninfected individuals. The mechanisms behind oxygen use in acanthocephalan-amphipod relationships is of  particular interest as hypoxia and anaerobic metabolism have been implicated as potential mechanisms by which \emph{P. minutus} alters host behaviour \citep{perrot2016}. The suppression of respiration seen in this study provides support to the theory that the parasite reduces host aerobic metabolism and provides some support to the theory that anaerobic metabolism may be involved in parasitic manipulation at temperatures below 15\degree C . 

Much importance has been placed on the trophic interactions between parasites and their hosts \citep{vanveen2008, chen2011, benesh2014}. However, I found no effects of parasitism on feeding rates, growth rates, rates of assimilation, or feeding efficiency measured within individual gammarids. Previous studies of the influence of parasites on amphipod feeding rates is equivocal, with some demonstrating significant impacts of acanthocephalan parasitism \citep{labaude2016, laverty2017} and others showing no effect \citep{fielding2003}. The specificity of host-parasite interactions may explain this ambiguity and may explain why I did not see an impact of parasitic infection on feeding rate, a finding consistent with feeding rates determined \emph{in situ} in Chapter \ref{chap:shannon}. The small impact of the parasite on energy flow is consistent with the assertion that parasite infection may not in itself be as energetically costly to host organisms as thought previously \citep{labaude2015influence}, a realisation with important implications for energy flow and models for future ecosystem management.

Our study focused on a single \textemdash though important and well-established \textemdash model system. While this limits somewhat the capacity to generalise our findings, this limitation is mitigated somewhat by the almost global presence of gammarid amphipods \citep{karaman1977, westram2011}, and their dominance of animal biomass in many communities \citep{mortensen1982}, where they process detritus, act as predators, and act as key energy conduits to higher trophic levels \citep{macneil1997, little2018}. Further work on the interactions between temperature and parasitic infection in additional model systems, particularly those that involve behaviour-modification in intermediate hosts, are needed to examine the generality of our results and explore further the potential impacts of warming on predator-prey interactions mediated by behaviour. 

By altering behaviour and therefore modifying predator-prey interactions, our results indicate an additional mechanism through which warming could have unforeseen consequences for the structure and dynamics of communities.  These findings highlight the need for holistic knowledge of interaction networks, incorporating both direct and indirect trophic and non-trophic interactions \citep{kefi2016}, to predict the effects of warming on the dynamics and stability of ecosystems \citep{kordas2017}. There remains a critical need for experiments in natural communities to test the generality of these predictions in a variety of ecological contexts and help us understand and anticipate how ecosystems will change in a warming world.

