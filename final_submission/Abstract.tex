\documentclass[a4paper,11pt]{article}

\usepackage{natbib}
\usepackage{enumerate}
\usepackage[osf]{mathpazo}
\usepackage{lastpage} 
\pagenumbering{gobble}
\linespread{1}
\date{}

\begin{document}

\begin{center}
%Title
\title{The importance of parasites in the functioning of warming ecosystems}
\maketitle
\date{ \vspace{-5ex}}


%Author
\noindent{Maureen Anne Williams}\\

\end{center}
%\section{Abstract}

Parasites play a critical role in the structure and functioning of ecosystems, contributing to overall ecosystem stability and the provision of ecosystem services. As the global climate changes, it is of considerable importance that ecologists understand and are able to predict shifts in the relationships between parasites and their hosts. Here, I utilize a model system comprising the parasite \emph{Polymorphus minutus} and its intermediate host, \emph{Gammarus duebeni}, in a combination of experiments done both in the laboratory and in the field, to explore how temperature moderates the effects of parasites on ecosystem structure and functioning. 

In chapter 2, I examine the impact of parasitic infection on the energy budget and behaviour of its host across a range of temperatures.Temperature was the major driver of energy flow within individual hosts, parasitic infection reduced host respiration at low temperature, and cystacanth-stage parasitic infection did not influence feeding rates or energy assimilation. However, warming increased the movement of hosts towards light and upwards in the water column, as did parasitic infection. In Chapter 3,  I present the first evidence that parasitism by \emph{P. minutus} increases the rates at which hosts dig into bottom sediments. I additionally show that temperature influences rates of bioturbation. In Chapter 4, I demonstrate experimentally the temperature-dependence of the specific components of handling time (pursuit time, capture time, subjugation time) in gammarid amphipods and show that the status of the potential prey items (mobile or sessile) determines the pattern of the temperature dependence of feeding times. Finally, I show that  \emph{in situ per capita} shredding rates were dependent on the size of the group of amphipods. Infection with \emph{P. minutus} and higher temperature increased survival rates in the field. 

My results show that both warming and parasitic infection alter the ecological role played by their host organisms. Moreover, they reveal that the drivers of host functioning are complex and interactive, with intraspecific competition, host sex, and even the mobility status of their prey influencing host behaviour. I conclude that warming and parasitism, through altering host behaviour and modifying predator-prey interactions, could have significant and unforeseen consequences for the structure and dynamics of ecosystems.

\end{document}
