\chapter{General Introduction}
\label{chap:intro}

%---------------------
%
% GENERAL INTRO
% 
%---------------------

\section{Parasites and Ecosystem Function}
Parasites are key contributors to both the biomass and biodiversity of ecosystems throughout the globe \citep{dobson2008,amundsen2009, jorge2018}. Every ecosystem hosts parasites, though even recent estimates of parasite biodiversity and distribution are likely underestimates due to patchy and idiosyncratic sampling, generally low sampling intensity, complex life stages of parasites which cycle through multiple hosts, and the frequency of cryptic species \citep{dallas2017, jorge2018, poulin2018the}. Theory suggests that around forty percent of the world’s species are parasitic for at least part of their life cycles \citep{dobson2008}. Full ecosystem characterizations are consistent with this \citep{lafferty2006parasites, lafferty2006food,kruis2008,grabner2017} and have revealed that the biomass of parasites is equivalent to that of other dominant consumers in ecosystems, including birds and fish \citep{kruis2008}. 

In spite of their important contributions to species richness and biomass in ecosystems, there is a dearth of empirical data on the contributions of parasites to ecosystem functioning. Parasitic interactions can comprise up to 75\% of all links within food webs \citep{sukhdeo2012,dunne2013}. Parasites are therefore likely of great importance to overall trophic dynamics \citep{hernandez2008,amundsen2009}, though their influence on the stability and functioning of ecosystems can be highly context-dependent \citep{benesh2008, perrot2014}. Accordingly, many studies have called for empirical investigation of the impact of parasites on ecosystem function and dynamics to improve our understanding of these crucial components of ecosystems \citep{blasco2017, carlson2017, vannatta2018}. This need is amplified even further by current global environmental change. Climate change, and its associated increased temperatures, increased frequency of disturbances, rising sea levels, and altered chemical cycling \citep{ipcc2014}, is predicted to have significant consequences for the structure of parasitic communities \citep{brooks2007, dobson2008}, including alterations in the geographical distribution of parasites and changes in the seasonality of parasite life cycles \citep{lafferty2009}. Up to thirty percent of parasitic worms are likely to  go extinct in the next fifty years, due to both extinction of their hosts and range shifts \citep{carlson2017}. Given that temperature is a major driver of within-host parasite dynamics \citep{kirk2018} and parasites play an important role in influencing the capacity of ecosystems to resist and recover from disturbances \citep{brooks2007}, this has profound implications for the stability of ecosystems globally.

Rising temperatures are likely to strongly influence biotic communities in aquatic ecosystems, as the majority of species are ectothermic and have little or no capacity to regulate their body temperature. The impact of warming on parasites, both as a possible mechanism of parasite release for their host species, and as a general ecological and evolutionary question, has been received little attention outside the scope of human parasites \citep{scott1984,gonzalez2010}. Only very recently have the intersections of parasitology and climate science in non-human, non-livestock animals begun to be explored, to-date with limited temperature ranges \citep{labaude2017} and a sole focus on trophic interactions \citep{laverty2017}. 

Many parasites impact their hosts directly, and behaviour-modifying parasites can have particularly strong effects on their hosts that can propagate through ecological networks \citep{labaude2015host}. Parasite-induced changes in host behaviour can have significant consequences for habitat structure \citep{mouritsen2005}, biogeochemical cycles \citep{vannatta2018} and energy flows \citep{sato2011}. Parasite-induced altered flows in energy are linked to changes in the composition of the community as a whole; parasites may provide an advantage to one competitor \citep{hatcher2014} or alter the overall preferences of consumers \citep{bernot2007,sato2011,bunke2015}, both of which lead to shifts in community structure and dynamics. These altered changes in community structure have even been linked to the persistence of top predators when the ecosystem is disturbed \citep{lefevre2009}. The impact of behaviour-changing parasites is not always mediated trophically, as behavioural-altering parasites have been linked to altered geographic distributions of their hosts \citep{frick2015}, and recent work suggests that behaviour changing parasites are even capable of altering the behaviour of non-hosts in ecosystems \citep{demandt2018}. 
Behaviour-modifying parasites are, therefore, a highly impactful group of species with major ecological consequences. Robust understanding of the influence of temperature on parasites and the relationships with their host organisms is, therefore, essential to predicting how parasites may modify the dynamics and stability of ecosystems under global environmental change. 

\section{Model System}

In order to study the impacts of behaviour-modifying parasites and temperature on ecosystem function, I experimentally examined a model system with an acanthocephalan parasite and its intermediate amphipod host, \emph{Gammarus duebeni}. Acanthocephalan parasites are macroparasites, meaning they can be seen with the naked eye, and infections can often be quickly identified while hosts are alive, making them an excellent model species for parasitic study. Acanthocephalan-amphipod relationships are found globally \citep{hynes1958, tokeson1982}, and \emph{Gammarus} are key detritivores in aquatic systems throughout the world, which speaks to their ecological relevance. Acanthocephalan parasites can impact their hosts in a broad variety of ways including: reductions of host immunocompetence \citep{rigaud2003},  increases in swimming speeds \citep{medoc2008}, reductions in female fecundity \citep{dezfuli1999}, castration of hosts \citep{bailly2017},  and increased vulnerability of hosts  to predation \citep{lagrue2007}. Acanthocephalan are among the few parasites have altered host behaviours in a manner that is adaptive to the survival of the parasite \citep{poulin1995}. The parasites reduce predator-avoidance of their intermediate hosts to predation by the parasites’ definitive hosts \citep{jacquin2014}. The magnitude of manipulation by acanthocephalans can be impacted by: (1) the age, sex, and size of the parasite; (2) host size and weight; and (3) parasite load, volume, and species composition \citep{labaude2015host}. Acanthocephalan parasites can also castrate their hosts \citep{kakizaki2003,bailly2017}, contributing to altered behaviour, altered flow of energy within hosts, and potential changes to population structure linked to parasite prevalence. 

The most common amphipod in Irish freshwaters is \emph{Gammarus duebeni} var. \emph{celticus} \citep{reid1938,macneil2009}, a known intermediate host of acanthocephalan parasites. The interactions between Irish amphipods and their acanthocephalan parasites have not been characterized within the last two decades outside of some specific, highly characterized, sites in the north of the country \citep{dick1993,lyndon1996,dunn1998}. In order to address this dearth of information, I conducted a survey on the presence of amphipod hosts in Irish river systems and lakes in 2015 (Fig. \ref{fig:acanth_map}). While \emph{Echinorhynchus truttae}, \emph{Polymorphus minutus}, and a single \emph{Pomphorhynchus laevis} were found in \emph{G. duebeni}, the most commonly identified parasite I found was \emph{Polymorphus minutus}. 

%------------------------------------------	

% Map of acanth infections across ireland
\begin{figure}[H] %!htb keeps the figure in this section before moving onto the discussion
	  \centering
	  \includegraphics[keepaspectratio,width=\textwidth]{figures/ch1/acanth_map.pdf}
	   \vspace*{-2.5cm}
	    \caption[Dominant Acanthocephalan parasites of \emph{Gammarus} spp.\ across Ireland] 
	    {Dominant acanthocephalan parasites of \emph{Gammarus} spp.\ across Ireland. Most sites sampled were either dominated by \emph{P.\ minutus} or had no infected \emph{Gammarus} spp.\ present.}
	  \label{fig:acanth_map}
	\end{figure}
	
%------------------------------------------	

\emph{P. minutus} is an acanthocephalan parasite that cycles through \emph{Gammarus} spp.\ as an intermediate host and utilizes water fowl as a final definitive host where the adults are able to reach reproductive maturity (Fig. \ref{fig:lifecycle}). Like many acanthocephalans, \emph{P. minutus} cystacanths have been shown to alter the behaviour of their amphipod hosts to increase the chance of parasite transmission to the definitive host \citep{jacquin2014, labaude2017}. 

%------------------------------------------	

%P minutus life cycle
\begin{figure}[H] %!htb keeps the figure in this section before moving onto the discussion
	  \centering
	  \includegraphics[keepaspectratio,width=\textwidth]{figures/ch1/pminutus_lifecycle.pdf}
	    \vspace*{-1.5cm}
	\caption[Life cycle of \emph{Polymorphus minutus} in Ireland] %This is the label in table of contents
	    {Life cycle of \emph{Polymorphus minutus} in Ireland. The parasite has one intermediate host, \emph{Gammarus duebeni}, and a definitive bird host, typically ducks or swans.}%this is under the figure
	  \label{fig:lifecycle}
	\end{figure}
	
%------------------------------------------	


A number of studies have established that \emph{P. minutus} manipulates the behaviour of \emph{Gammarus} \citep{kaldonski2008, perrot2016}. \emph{P. minutus} reduces the fecundity of amphipods \citep{dezfuli1999}, increases movement upwards in the water column \citep{perrot2016}, and reduces host activity levels \citep{jacquin2014}. The manipulative activity of \emph{P. minutus} is sensitive to environmental factors \citep{perrot2016}, with the oxygenation of the environment influencing the parasite’s ability to alter behaviour. The ability of the parasite to alter behaviour also depends on the life-stage of the parasite, as the cystacanth stage changes behaviour but the acanthella stage of the parasite does not \citep{bailly2017}. Few studies have addressed the crucial interaction between parasitic infection and temperature in controlling the behaviour of gammarid amphipods, especially across broad temperature ranges similar to those experienced seasonally in temperate ecosystems and those expanded temperature ranges that are predicted under current climate change models \citep{labaude2016, labaude2017}. The ways in which this manipulation impacts whole-ecosystem functioning remains unknown. This thesis explores the impacts of behaviour-manipulating acanthocephalan parasites and increasing temperatures on the functioning of aquatic ecosystems. By examining host physiology, trophic ecology, behavioural ecology, and even ecosystem engineering in aquatic ecosystems, this thesis will paint a fuller picture of the role of parasites in a warming world. 

\section{Structure of this Thesis}

The aim of this thesis is to explore the independent and combined effects of behaviour-modifying parasites and temperature on the structure and functioning of ecosystems. To this end, I conducted experiments both in the laboratory and in the field to examine the influence parasites have on their hosts at multiple temperatures on the following aspects of ecosystem and host function: 

\begin{enumerate}
\item	Host behaviour and energy flow at the individual level (Chapter 2); 
\item	Ecosystem engineering via bioturbation  (Chapter 3); 
\item	Trophic ecology and host feeding preferences  (Chapter 4); 
\item	Host feeding and ecosystem function in the field (Chapter 5).
\end{enumerate}

\subsection{Chapter 2}
A thorough knowledge of the influence of parasites on their individual hosts is a crucial first step towards understanding how parasites can impact overall ecosystems, particularly in light of warming temperatures. In this chapter, I examine the impact of parasitic infection on the energy budget and behaviour of its host across a range of temperatures. My results show that a temperature increase of six degrees is enough to modify the predator avoidance behaviour of hosts to the same extent as infection with \emph{P. minutus}. Temperature was the major driver of energy flow within individual hosts, parasitic infection reduced host respiration at low temperature, and cystacanth-stage parasitic infection did not influence feeding rates or energy assimilation. However, warming increased the movement of hosts towards light and upwards in the water column, as did parasitic infection. Parasitic infection and increasing temperature additively impacted anti-predator behaviour, increasing the hosts' vulnerability to predation. As gammarid amphipods are crucial to overall ecosystem function, this altered vulnerability to predation could lead to shifts in energy flow through the ecosystem. As infected amphipods and amphipods at higher temperatures moved upwards in the water column, I predicted that the infected amphipods would have less interaction with bottom sediments, though this had not yet been examined experimentally. 

\subsection{Chapter 3}
It has been suggested that parasites might act as ecosystem engineers as parasites can control the behaviour of animals that physically mediate habitat structure. There are a number of processes in which parasites may exert their influence within their environments. In aquatic ecosystems, one of the most critical animal-mediated processes is the process of bioturbation, in which animals dig into the sediment, releasing nutrients and aerating the bottom sediments. Gammarid amphipods are active biotubators, particularly within the surficial levels of the sediments. In this chapter, I present the first evidence that parasitism by \emph{P. minutus} increases the rates at which hosts dig into bottom sediments. I additionally show that temperature influences rates of bioturbation. Temperature and parasitic infection combined additively, with no evidence for antagonistic or synergistic interactions found.  This is the first record of a parasite increasing bioturbation activity of a host, and the first time \emph{P. minutus} has been shown to influence this behaviour. Increased rates of bioturbation are likely to influence the concentration of soluble nutrients in the water and could potentially alter the levels of oxygenation in the sediments of the river bed, which could reduce anaerobic respiration and denitrification levels in the ecosystem.  

\subsection{Chapter 4}
While there are many types of interactions between organisms, trophic interactions are by far the most studied. The temperature-dependence of the specific components of handling time and attack rates in omnivorous, ectothermic animals, such as gammarid amphipods, can reveal crucial information about both the mechanisms that underlie the structure of food webs and the relative strength of those interactions across temperature ranges. Parasites have the potential to alter the patterns of trophic interactions in ecosystems, as they alter the behaviour of their hosts. In this chapter, I demonstrate experimentally the temperature-dependence of the specific components of handling time (pursuit time, capture time, subjugation time) in gammarid amphipods and show that the status of the potential prey items (mobile or sessile) determines the pattern of the temperature dependence of feeding times.  Gammarids did not display preferences for potential prey items initially, though the amphipods were more likely to switch items between pursuit and capture when prey were mobile.  These data provide a thorough and direct examination of the ability of amphipods to capture their prey linked to information about preference. As temperatures increase, we predict that we will see higher levels of scavenging and shredding in gammarid amphipods. 

\subsection{Chapter 5}
The impacts of parasites and increasing temperature on the function of their ecologically important hosts were examined in the laboratory in the prior chapters. There is, however, a dearth of information on how realistically these results apply to actual ecosystems. To achieve more realistic estimates of the impact of warming and parasitism on hosts, I examine the influence of parasitic infection, group size of the host, and temperature on the feeding rates of \emph{G. duebeni in situ} in this chapter. I set up decomposition bags with set numbers of infected or uninfected individuals. These bags were placed in experimentally warmed and ambient areas of the River Shannon, and the higher temperature treatment was provided by the thermal effluent generated by a power plant along the River Shannon. I show that per capita shredding rates were dependent on the size of the group of amphipods. Infection with \emph{P. minutus} and higher temperature increased survival rates. The experiment was repeated in winter and summer, allowing us to show that most trends observed in winter were consistent over seasons. These findings show that intraspecific competition, temperature, and infection status contribute to the structure of the amphipod population in real ecosystems and the function of gammarid amphipods in their environment. 

\subsection{Chapter 6}
The final chapter of this thesis focuses on the overall consequences of warming and parasitic infection on the function of aquatic ecosystems. Here I discuss the importance of acanthocephalan parasites as potential ecosystem engineers by summarizing the results of the experiments presented in the previous chapters. I highlight the need for experimental and observational data on the impacts of increasing temperature for the host-parasite interaction and suggest how future research may continue to examine the indirect impacts of parasites in their environments. 

%\bibliography{References}