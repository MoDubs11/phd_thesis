\chapter*{Summary}
\chaptermark{summary}
\addcontentsline{toc}{chapter}{Summary}

Parasites play a critical role in the structure and functioning of ecosystems, contributing to overall ecosystem stability and the provision of ecosystem services. As the global climate changes, it is of considerable importance that ecologists understand and are able to predict shifts in the relationships between parasites and their hosts. Here, I utilize a model system comprising the acanthocephalan parasite \emph{Polymorphus minutus} and its intermediate host, \emph{Gammarus duebeni}, in a combination of laboratory and field experiments, to explore how temperature moderates the effects of parasites on ecosystem structure and functioning. My results show that both warming and parasitic infection alter the ecological role played by host organisms. Moreover, these results reveal that the drivers of host functioning are complex and interactive, with intraspecific competition, host sex, and even the mobility status of their prey influencing host behaviour. I conclude that warming and parasitism, through altering host behaviour and modifying predator-prey interactions, could have significant and unforeseen consequences for the structure and dynamics of ecosystems.