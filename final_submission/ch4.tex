\chapter[The temperature-dependence and influence of prey mobility on multiple components of handling time]{The temperature-dependence and influence of prey mobility on multiple components of handling time}
\label{chap:feeding}

\section{Summary}
Species are known to respond asymmetrically to warming, potentially altering the links between parasites and their hosts as the global climate warms. The ability of omnivores to capture and process their food can be altered by both parasitic infection and changes in temperature, though the combined impacts of these drivers have yet to be fully examined empirically. Here, I divide handling time into its components (time in pursuit, time to capture, and time to subjugate potential prey) to determine whether each handling time process, and the potential prey chosen at each point, respond similarly to temperature changes and whether parasites and the mobility status of prey could alter those thermal dependencies. I found that each component of handling time had unique temperature-dependencies, though the overall propensity of amphipods to eat was increased at higher temperatures. The mobility of potential prey had significant impacts on the time it took gammarids to capture and subjugate their food. Parasites only influenced the time it took predators to capture prey, suggesting that the impact of larval acanthocephalans is limited to a very specific process within predation. These results highlight the complexity of the interactions between predators and prey, whereby each process within the interaction is controlled by different drivers and thus is likely subject to different limitations. As temperature increases and the movement behaviour of prey species change, these findings highlight the need to consider the multiple distinct components of predator-prey interactions when predicting shifts in energy flow and ecosystem stability under climate warming.


\section{Introduction}

The nature and strength of trophic interactions are a key determinant of the stability of ecosystems \citep{may1974, pimm1984, ogorman2009, allesina2012, donohue2016, donohue2017, delong2018}. The consequences of a warming climate change for consumer-resource interactions depend to a large extent upon the thermal optima of individual species, with some species gaining advantages over others due to asymmetries in performance \citep{dell2014, penk2016}. Climate warming is predicted to have particularly profound consequences for ectothermic animals, as it will increase the metabolic demand of organisms in general, likely resulting in significant shifts in both intra- and interspecific interactions and altered habitat use \citep{kordas2011, abram2017, vandervorste2017}. 

The metabolic theory of ecology (MTE) \citep{brown2004} posits that ecological dynamics across all levels of organisation are broadly controlled by the temperature-dependence of metabolism. The theory has been used both to understand how ecosystems function \citep{ohlund2015, sentis2015, ogorman2017} and predict how they will change as temperatures increase \citep{oconnor2009, petchey2010}. Linking trophic relationships into the MTE framework is challenging, however, due to the existence of threshold temperatures below which consumers are inactive \citep{ohlund2015}, the importance of spatial and temporal scale \citep{kordas2011}, and the relative advantage of prey items at lower temperatures due to smaller size \citep{dell2011}.

Handling time --- the time it takes for predators to chase, capture, process, and digest their prey \citep{jeschke2002} --- plays a critical role within the MTE, by providing valuable information for predicting predator behaviour and interpreting the behavioural response of both predators and their prey \citep{rall2012, reuman2014}. The many steps that make up handling time can, however, each be subject to different drivers and respond to elevated temperatures in different ways \citep{sentis2013}, as animals are more likely to expend energy on certain tasks over others \citep{dell2011}. Explicit quantification of these various components, such as attack rates \citep{ohlund2015}, digestion \citep{sentis2013} and excretion rates \citep{rall2010, enlund2011}, has helped to explain the deviation of many observed measurements from those predicted by theory. Mechanistic understanding of the thermal-dependence of the distinct steps that make up handling time is, therefore, critical for the prediction of shifts in trophic interactions under global climate change. 

Aside from its moderation of handling times, temperature may also influence the feeding choices of predators \citep{ferrari2015}, particularly when they are omnivorous  \citep{boersma2016}. At higher temperatures, predators tend to become more selective, aiming for higher quality food \citep{gordon2018}. Though this has potentially profound implications for energy transfer and the stability of food webs \citep{ogorman2016}, no empirical studies have examined the influence of temperature on both consumer food choice and the various components of prey handling times simultaneously. 

Here, I examine experimentally the thermal-dependence of the specific components of handling time and food choice of \emph{Gammarus}, a key omnivorous ectotherm in aquatic ecosystems. Infection with parasites is known to influence the feeding preferences of \emph{Gammarus} \citep{bunke2015} and the rate at which they process food material \citep{labaude2016}. Given parasites' ubiquity and ecological importance in ecosystems globally \citep{poulin2018the}, I examined whether, and how, (1) temperature and parasitism, in isolation or in combination, influence the various components of handling time and / or feeding preferences of \emph{Gammarus} and (2) predatory and scavenging behaviour are moderated similarly by temperature and / or parasitic infection. 

To address these questions, I utilize an established model system, the amphipod \emph{G. duebeni celticus} and its acanthocephalan parasite \emph{P. minutus}. \emph{G. duebeni} is a dominant benthic macroinvertebrate in many freshwater ecosystems \citep{mortensen1982, kelly2006}, as it plays an important role in ecosystem functioning \citep{macneil1997, sutcliffe2000}. The parasite \emph{P. minutus} is relatively well-studied \citep{dezfuli1999, bailly2017}, and the cystacanth-stage larvae of the parasite is a known manipulator of host behaviour \citep{jacquin2014}, is easily identified in live hosts, and is common across Europe and North America \citep{vancleave1951}. 


\section{Method}

\subsection{Organism collection and preparation}

Adult \emph{G. duebeni} used in the experiment were collected from Lough Lene, Co. Westmeath, Ireland (53\degree 39'37.1"N 7\degree 11'41.1"W) along with juvenile \emph{G. duebeni} individuals that were used as potential prey. \emph{Asellus aquaticus} were collected simultaneously from nearby Lough Ennell, Co. Westmeath, Ireland (53\degree 28'13.2"N 7\degree 23'00.2"W). \emph{P. minutus}-infected and uninfected adult \emph{G. duebeni} were placed in individual housings and maintained at 15\degree C  with a 12h:12h light:dark cycle. For all trials, \emph{G. duebeni} were acclimated to the experimental temperatures with 3\degree C  temperature changes per day \citep{penk2016}.

Alder (\emph{Alnus glutinosa}) leaves were cut into 0.7 mm discs and conditioned for 72 hours in Lough Lene water to improve palatability. Potential prey items --- \emph{A. aquaticus} and juvenile \emph{G. duebeni} --- were weighed individually before being presented live for predation trials. For scavenging trials, prey items were euthanized by freezing and were defrosted fully before use. 

\subsection{Trial apparatus}

Trials were conducted in a purpose-built experimental chamber (Fig. \ref{fig:chamber}). The main body of the chamber is a petri dish, which was darkened with a black marker on the outside of the dish in order to reduce the stress of the experimental animals. In the centre of the dish, a “pizza saver” was used to provide a refuge. This refuge allows the experimental individual to hide, in order to separate leaf-feeding behaviour from the tendency of the animals to use leaves as a refuge. From each of the three points of the pizza saver, a glass slide cover was used to divide the petri dish into three areas, each of which the adult \emph{G. duebeni} could access. For trials, one of the three potential prey items was placed in each chamber. Observations confirmed that prey items did not leave their chamber or interact with the other prey options. Water in the dish was changed between trials and the location of each prey item was varied haphazardly among trials.

\begin{figure}[H]
    \centering
    \includegraphics[scale=0.6]{figures/ch4/chamber.pdf}
    \vspace*{-1.5cm}
  \caption [Trial apparatus designed for \emph{G. duebeni} feeding preference trials]{The design of the three chamber apparatus for \emph{G. duebeni} feeding preference trials with the refuge in the centre.} 
    \label{fig:chamber}
\end{figure}


\subsection{Experimental design}

Trials were run at 3\degree C intervals for temperatures between 6\degree C and 21\degree C, as this encompasses both the majority of the typical temperature range of lakes in Ireland and includes a temperature based on future climate change scenarios in Irish lakes (21\degree C; \cite{ipcc2014}). Adult mortality at temperatures exceeding 21\degree C was greater than 50\% and thus we limited trials to a maximum of 21\degree C. 

Adult \emph{G. duebeni} were allowed five minutes to acclimate to the experimental housing, which was filled with clean, oxygenated water at the same temperature as the animals' housing. After acclimation, the amphipod was encouraged into the hide, and then presented with a leaf disc, juvenile \emph{G. duebeni} and \emph{A. aquaticus} in each of the three food chambers. The latter two organisms were presented live in predation trials and dead in scavenging trials. The identity and time of the adult’s first touch of a prey item, the time and item of first capture, where the adult grasped hold of a prey item, and the time and item of the adult’s first bite were all recorded. Trials ended after 10 minutes of observation. The time taken for an adult to touch an item was characterized as the pursuit time, the time between touch and capture as capture time, and the time between capture and first bite as subjugation time. 

Following trials, adult \emph{G. duebeni} were weighed and dissected to ensure infection status as per methods in Chapter \ref{chap:physbeh}. For all analyses, only adult \emph{G. duebeni} infected with single cystacanth-stage infections of \emph{P. minutus} were included in analyses as infected and only \emph{G. duebeni} adults with no acanthocephalan infection were included as uninfected. 

\subsection{Data analyses}

The time predators spent in pursuit, time it took for predators to capture prey, and time predators spent subjugating prey were analysed using generalized linear mixed models (GLMMs) with Gamma distributions and an inverse link using lme4::glmer \citep{bates2015}. In these analyses, time (in seconds) was the response variable, gammarid individual was a random effect, temperature, infection status, and feeding mode (i.e.\ availability of live or dead prey) were fixed effects. All possible interaction terms were included in initial analyses. The items touched, captured, and bitten were analysed as GLMMs using MCMCglmm \citep{hadfield2010} with multinomial distributions. For each of these analyses, temperature, infection status, and prey status were fixed effects, gammarid individual was random, and prey item (i.e.\ juvenile \emph{G. duebeni}, \emph{A. aquaticus}, leaf, no response) was the response variable. The proportion of adults eating, defined as whether or not an adult bit an item, was analysed using a GLMM with a binomial distribution modelled with the lme4 package. In these analyses, whether or not the individual ate was a binary response variable, gammarid individual was random, and infection status, prey status, and temperature were fixed effects. Finally, the number of adult \emph{G. duebeni} that switched items between initial touching and biting was analysed with a GLMM with a binomial distribution with lme4::glmer and gammarid individual as a random effect, and with infection status, prey status, and temperature as fixed effects. All data analyses were performed in R version 3.4.1 \citep{r2017}.

\section{Results}

The effect of temperature on the likelihood of \emph{G. duebeni} feeding varied depending upon the status of their prey (i.e.\ there was a significant interaction between temperature and whether their potential prey were dead or alive; Table \ref{tab:feedtab}a; Fig. \ref{fig:propeat}). The likelihood of feeding increased significantly with temperature when available prey were dead, whereas temperature had no effect on the likelihood of feeding when live prey items were present (Fig. \ref{fig:propeat}). 



\begin{landscape}
\begin{table}
\caption[The effects of temperature, parasitic infection, and prey mobility status on the feeding behaviour of  \emph{G. duebeni}.]{Effects of temperature, infection and prey status on \emph{G. duebeni} feeding behaviour. Significant (\emph{P} $<$ 0.05) effects are highlighted in bold.}
\label{tab:feedtab}
\begin{tabular}{llllllllllllllll} \toprule
 & Response & \multicolumn{2}{l}{Temperature} & \multicolumn{2}{l}{Infection} & \multicolumn{2}{l}{Prey Status} & \multicolumn{2}{l}{Temp$\times$} & \multicolumn{2}{l}{Temp} & \multicolumn{2}{l}{Infection$\times$} & \multicolumn{2}{l}{Temp$\times$Infection} \\
 &  & \multicolumn{2}{l}{} & \multicolumn{2}{l}{} & \multicolumn{2}{l}{} & \multicolumn{2}{l}{Infection} & \multicolumn{2}{l}{Prey Status} & \multicolumn{2}{l}{PreyStatus} & \multicolumn{2}{l}{$\times$Prey Status} \\ \midrule
 &   & \emph{t/z} & \emph{p} &\emph{t/z} & \emph{p} & \emph{t/z} & \emph{p} & \emph{t/z} & \emph{p} & \emph{t/z} & \emph{p} & \emph{t/z} & \emph{p} & \emph{t/z} & \emph{p} \\ 
a & Number Eating & \textbf{4.61} & \textbf{$<$0.01} & -1.20 & 0.26 & 0.34 & 0.73 & -0.93 & 0.35 & \textbf{-2.53} & \textbf{0.01} & 0.62 & 0.54 & 0.89 & 0.38 \\
b & Pursuit Time & \textbf{2.08} & \textbf{0.04} & -0.90 & 0.37 & -0.48 & 0.63 & 0.69 & 0.49 & -0.44 & 0.66 & 1.36 & 0.17 & -0.63 & 0.53 \\
c & Capture Time & 0.40 & 0.53 & 0.01 & 0.93 & \textbf{8.91} & \textbf{$<$0.01} & 2.11 & 0.15 & 0.13 & 0.72 & 1.77 & 0.18 & \textbf{6.32} & \textbf{0.01} \\
d & Subjugation Time & <0.01 & 0.99 & 0.22 & 0.64 & \textbf{4.08} & \textbf{0.04} & 0.22 & 0.64 & 0.07 & 0.80 & 0.05 & 0.82 & 0.22 & 0.64 \\
e & Pursuit Item &  & 0.40 &  & 0.17 &  & 0.20 &  & 0.20 &  & 0.47 &  & 0.72 &  & 0.44 \\
f & Capture Item  &  & 0.49 &  & 0.12 &  & \textbf{$<$0.01} &  & 0.26 &  & 0.56 &  & 0.49 &  & 0.44 \\
g & Subjugation Item &  & 0.60 &  & 0.18 &  & \textbf{$<$0.01} &  & 0.31 &  & 0.59 &  & 0.94 &  & 0.80 \\
h & Switch Item & -0.38 & 0.70 & 1.53 & 0.13 & \textbf{-2.70} & \textbf{$<$0.01} & -1.50 & 0.13 & 0.12 & 0.90 & -1.13 & 0.26 & 1.36 & 0.17 \\ \bottomrule
\end{tabular}
\end{table}
\end{landscape}

\begin{figure}[H]
    \centering
    \includegraphics[scale=0.7]{figures/ch4/propeat.pdf}
  \caption [Proportion of \emph{G. duebeni} eating across a range of temperatures when prey were mobile or sessile]{Proportion of trials at each temperature where \emph{G. duebeni} individuals ate or did not eat when amphipods were pursuing dead (scavenging) and live prey (predating).} 
    \label{fig:propeat}
\end{figure}

\subsection{Components of handling time}

The specific components of handling time varied significantly in their response to temperature and prey status (Table \ref{tab:feedtab}b-d). Temperature was the only driver of amphipod pursuit time (Table \ref{tab:feedtab}b; Fig. \ref{fig:feedtime}a), with pursuit times decreasing as temperatures increased regardless of the prey being pursued (Fig. \ref{fig:feedtime}a). Time to capture prey was determined by a three-way interaction among temperature, parasitic infection, and potential prey status (Table \ref{tab:feedtab}c; Fig. \ref{fig:feedtime}b). Parasitic infection increased capture times when amphipods were actively predating and at lower temperatures when amphipods were scavenging, though the effect of parasitic infection was not seen when adults were scavenging at higher temperatures (Fig. \ref{fig:feedtime}b). In contrast, temperature had no effect on the time adults spent subjugating their prey (Table \ref{tab:feedtab}d; Fig. \ref{fig:feedtime}c). There was, however, a significant difference in subjugation time between when potential prey was alive or dead, with the amphipods spending significantly longer subjugating live compared with dead prey (Table \ref{tab:feedtab}d; Fig. \ref{fig:feedtime}c). Parasitic infection had no effect on pursuit or subjugation times of the amphipods used in the experiment (Table \ref{tab:feedtab}).

\begin{figure}[H]
    \centering
    \includegraphics[keepaspectratio,width=\textwidth]{figures/ch4/feedtime_1.pdf}
  \caption [Thermal dependence of components of handling time when potential prey were either mobile or sessile]{Thermal dependence of individual components of handling time when potential prey were either alive (predating) or dead (scavenging). (a) Pursuit time, (b) time to capture, and (c) time to subjugation. All times are shown in seconds and points are mean  $\pm$ S.E.} 
    \label{fig:feedtime}
\end{figure}

\subsection{Feeding preferences}

Amphipods did not show a preference for any of the available food items during their initial pursuit, regardless of host infection, temperature, or whether potential prey was alive or dead (Table \ref{tab:feedtab}e; Fig. \ref{fig:feedpref}a). However, the status of their potential prey as dead or alive determined its likelihood of capture (Table \ref{tab:feedtab}f; Fig. \ref{fig:feedpref}b). When only dead or inactive prey were available, adults were equally likely to capture and subjugate any of the food items present. However, when predation was possible, amphipods were significantly more likely to capture and subjugate leaf matter than either of potential live prey available (Table \ref{tab:feedtab}f,g; Fig. \ref{fig:feedpref}). Finally, the proportion of individuals switching prey items between initial pursuit and actual capture was also influenced by the status of the potential prey, with significantly more adults switching items when potential prey were alive than when they were not (Table \ref{tab:feedtab}f). Neither temperature nor parasitic infection had any effect on choice of prey. 

\begin{figure}[H]
    \centering
    \includegraphics[keepaspectratio,width=\textwidth]{figures/ch4/feedpref_1.pdf}
  \caption [Proportional predator preferences while potential prey items were alive (predating) or dead (scavenging).]{Proportional predator preferences while potential prey items were alive (predating) or dead (scavenging). The proportion of times each potential prey item was initially (a) pursued, (b) captured, and (c) subjugated by amphipod individuals.} 
    \label{fig:feedpref}
\end{figure}

\section{Discussion}

My results demonstrate clearly that the various components of handling time differ significantly in their responses to temperature and, moreover, that these patterns can vary depending upon whether potential prey were alive or dead. In contrast, food choice was not influenced by temperature, though amphipods generally avoided capturing and subjugating moving prey at all temperatures, while infection with behaviour-manipulating parasites had a somewhat surprisingly minor influence on handling time. These findings highlight the significant challenges associated with predicting how predator-prey interactions, and thus the stability of ecosystems, will change as the global climate warms.

Handling time is often quantified effectively as a one-dimensional process \citep{jeschke2002}. My results, in concert with previous studies \citep{jeschke2002, sentis2013}, challenge the assumptions made when grouping multiple behaviours, with different regulating factors, into a single metric \citep{enlund2011, sentis2013}. By parsing handling time into its constituent components, I show that pursuit, capture, and subjugation do not respond uniformly to warming. The presence of mobile live prey led to significantly longer capture and subjugation times compared to when they were dead or inactive. When prey were sessile, time in pursuit was faster at higher temperatures. However, in the presence of active potential prey, pursuit times were not affected by temperature, at least across the range of temperatures I examined. This is likely due to asymmetries in the performance of the comparably larger predators and their smaller prey \citep{dell2014}.  

The thermal responses of an organism can vary significantly among its physiological and ecological traits \citep{dell2011}. The thermal dependence of the propensity of gammarid amphipods to feed differed in the presence of live and mobile potential prey compared to dead, sessile prey. When prey were sessile and predators were scavenging, the proportion of \emph{Gammarus} that were feeding increased with temperature, as would be expected with increasing metabolic rates \citep{brown2004, schmidlin2015, labaude2016}.  In contrast, temperature did not affect the propensity to feed in the presence of motile prey. The dynamics of predators and their prey often do not scale linearly when both species are mobile \citep{dell2014} and, though I did expect to find a difference between the temperature dependence of handling time components when prey were motile compared to when they were sessile, I found no interactions between temperature and prey status in any of the components of handling time we examined. 

Prior work on ectotherms from arctic ecosystems has found thresholds in the feeding behaviour of some predators, where they would not attack below specific temperatures \citep{ohlund2015}. I found no evidence of the existence of such thresholds, possibly due to the fact that this study focused on organisms from seasonal temperate ecosystems. Nonetheless, these results highlight the difficulties associated with predicting the effects of warming on predator-prey dynamics, particularly as the effects of temperature are context-specific and exhibit considerable variation across systems \citep{enlund2011, ohlund2015}. 

Gammarids did not display a preference for the item they first pursued, and were equally likely to initially pursue juvenile conspecifics, other invertebrate prey, or leaf matter. However, capture and subjugation of live prey proved difficult, with gammarids showing a strong preference across all temperatures to capture and subjugate leaf matter in the presence of mobile prey. Many gammarids initially pursued live prey items before actually capturing and biting the leaf matter presented. This contributed to the increased times to capture in the treatments that included active prey. Research on preferences of aquatic macroinvertebrates in the field has found higher selectivity of some animals at higher temperatures, though the trend is species-specific \citep{boersma2016, ogorman2016}. The gammarids in my study did not, however, demonstrate a shift in preference at higher temperatures, and are consistent with the overall classification of gammarid amphipods primarily as shredders \citep{cummins1979, macneil1997}, with a secondary role as omnivorous scavengers.   

Surprisingly, the only component of feeding behaviour in my study that was influenced by infection with \emph{P. minutus} parasites was capture time. As previous work has found that parasitic infection can impact host functional response curves \citep{toscano2014} and host feeding preferences \citep{bunke2015}, I expected to find a significant role of parasitic infection in most of the response variables we examined. However, the role of parasites in the handling time of hosts was limited to generally increasing capture time, suggesting that the impact of infection with larval acanthocephalans is quite specific. Host-parasite interactions can exhibit a high level of specificity \citep{hynes1958, zittel2018}, and further experimental work exploring the generality of the impact of parasites on handling time and feeding preference is needed. 

Temperature plays an important role in the feeding behaviour of many ectothermic species and, as animals attempt to cope with increasing temperatures, there are a number of ways organisms can mediate the impact of temperature by altering their behaviour \citep{barnes2015, vandervorste2017}. Moreover, the response of organisms to warming is both behaviour- and species-specific \citep{dell2011, enlund2011, ohlund2015, kenna2017}. Even though gammarid amphipods are globally distributed \citep{karaman1977} and ecologically important \citep{little2018}, our study was limited to a single model system. Nonetheless, by splitting feeding behaviour into its individual components, our findings demonstrate how the specific components of handling time exhibit different thermal dependencies and drivers. Temperature, potential prey status, and parasitic infection impacted specific feeding behaviours, but frequently not additively. As climate warms, increasing temperatures have the potential to alter predator-prey dynamics significantly, and our findings highlight the need to consider the multiple distinct components of predator-prey interactions when predicting shifts in energy flow and ecosystem stability under climate warming. 