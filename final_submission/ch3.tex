\chapter[Infection with behaviour-manipulating parasites enhances bioturbation by key aquatic detritivores]{Infection with behaviour-manipulating parasites enhances bioturbation by key aquatic detritivores}
\label{chap:bioturb}

\section{Summary}
The ubiquity of parasites within their ecosystems and their potential impacts on host behaviour has led to the suggestion that parasites may act as ecosystem engineers, structuring their environment and physical habitats. Climate change may alter the relationships between hosts and parasites, potentially changing the rates at which hosts interact with their larger environment. Gammarid amphipods contribute to ecosystem function in aquatic environments by bioturbating, or digging into sediments, with consequences for sediment and water oxygenation and nutrient cycling within rivers and lakes. In this study, I test how temperature and infection with acanthocephalan parasites alters the rates at which \emph{Gammarus duebeni} rework surface sediments. I show that higher temperatures and parasitic infection additively increase rates of amphipod bioturbation. These increases in bioturbation will have knock-on effects for overall ecosystem functioning in a warming world. In terms of benthic habitat structure, parasites do seem to act as ecosystem engineers and future ecosystem management strategies should account for parasitic infection when predicting the impacts of increasing temperature.

\section{Introduction}

Parasites are found in all ecosystems throughout the globe \citep{jorge2018}. They comprise up to 40\% of all described species \citep{dobson2008}, feature in up to 70\% of the links within food webs \citep{dunne2013}, and contribute significantly to the biomass of many ecosystems \citep{kruis2008}. Their presence has, therefore, important --- though still remarkably underappreciated --- implications for the structure, functioning and dynamics of entire ecosystems \citep{amundsen2009, dunne2013}. However, their influence on how ecosystems respond to environmental change, particularly a warming climate \citep{kutz2005, hoberg2007}, remains largely unknown. Climate warming will likely modify rates of parasite transmission \citep{mouritsen1997} and infectivity \citep{studer2010}. Moreover, temperature also moderates host behaviour \citep{issartel2005, abram2017}, which can influence host susceptibility to parasites \citep{morley2014} and overall host functioning in the ecosystem \citep{ogorman2012}.

Bioturbation, the mixing of sediment by mobile organisms, is an important ecosystem function that occurs in both terrestrial and aquatic environments. Bioturbation comprises a key non-trophic mechanism through which organisms physically, chemically, and biologically structure ecosystems \citep{grant1994, Jones1996, baranov2016, wohlgemuth2017}. In aquatic ecosystems, bioturbation influences the flow of nutrients \citep{mermillod2006}, oxygenation of sediments \citep{baranov2016}, the turbidity of the water \citep{croel2011}, and sediment erosion rates \citep{grant1994}. Moreover, the rate of bioturbation in has also shown to increase with warming \citep{baranov2016}. There is, however, little known about the influence of parasites on rates of bioturbation \citep{vannatta2018}. While parasitism has been shown to modify burrowing behaviour in intertidal cockles \citep{mouritsen2005}, by reducing their digging into the sediments, there have been no cases where the presence of parasites increases bioturbation rates of their hosts. 

Gammarid amphipods contribute significantly to the bioturbation in aquatic ecosystems globally \citep{mermillod2006, hunting2012, denadai2013, vadher2015} by reworking the uppermost layer (i.e.\ 2-3 cm) of sediment. In freshwaters, gammarids are also frequently infected with an acanthocephalan parasite,\emph{ P. minutus}, which alters host movement in the water column and the rates at which they shred detritus \citep{bauer2005, labaude2016}. Two life-stages of the acanthocephalan, the acanthella and the cystacanth, utilize the amphipod intermediate host. The cystacanth is the life-stage associated most strongly with behavioural changes \citep{bailly2017}, as it is the stage at which the parasite is infective to its definitive, or final host. 

I examined bioturbation activity of \emph{G. duebeni} var. \emph{celticus} experimentally in the laboratory across a range of temperatures encountered in their native range to determine whether (1) parasitic infection and temperature, individually or in combination, modify rates of sediment surface reworking (our measure of bioturbation) by \emph{G. duebeni}, and (2) the combined effects of parasitic infection and temperature on bioturbation are additive, antagonistic, or synergistic. \emph{Gammarus} are used frequently as a model system to examine the impacts of parasites on intermediate host behaviour \citep{agatz2014, perrot2014, perrot2016} and \emph{G. duebeni} comprise important components of the benthic community throughout Ireland \citep{reid1938, macneil2009}, playing a crucial role in ecosystem functioning by processing of detritus \citep{kelly2002}. As the amphipods are ectothermic \citep{baranov2016}, I expect temperature to increase rates of bioturbation. I also predict that parasitic infection will reduce rates of bioturbation due to reduced interaction between the gammarids and the benthos as gammarid hosts infected with \emph{P. minutus} display enhanced phototaxis and are more likely to move upward in the water column \citep[Chapter \ref{chap:physbeh}]{perrot2016}. 


\section{Method}

\subsection{Experimental design}

I quantified the rate of sediment surface re-working by adult \emph{G. duebeni} at two levels of infection (i.e.\ infected or uninfected by \emph{P. minutus} cystacanths) and at four temperatures (4\degree C, 9\degree C, 14\degree C, and 19\degree C), encompassing the majority of the temperature range experienced by \emph{G. duebeni} in their native range in Ireland, in a full-factorial experiment. Each experimental treatment combination was replicated 20 times.

Amphipods, benthic lake sediments and lake water used in the experiment were collected from Lough Lene, Co. Westmeath, Ireland (53.6625\degree N, 7.234\degree W) on 22 January 2018. Surficial (i.e.\ less than 3 cm depth) benthic lake sediments were collected, homogenized, passed through a 1 mm sieve to remove macrofauna and rocks, and allowed to settle in lake water for a day before use.  

Bioturbation was quantified based upon methods developed by \citet{denadai2013} and \citet{wohlgemuth2017}. Eight 10 L buckets were filled with lake sediments to a depth of 5 cm. Sterile centrifuge tubes (85 mm long with an internal diameter of 2.7 cm) with their tops and bottoms removed were placed into the buckets (25 pipes per bucket). Tracer sand (pink luminophores $<$ 125 mm; Brianclegg Ltd., UK) was added to a depth of 0.2 cm within each tube. Filtered, aerated lake water was then added to the bucket to a depth of 13 cm above the sediment. A single \emph{G. duebeni} adult ($>$ 0.02 g fresh weight) was added to individual tubes which were then covered with mesh (hole diameter 1mm) to maintain the organisms within the tubes while allowing the circulation of aerated water. Each 8 L bucket contained ten tubes containing infected \emph{G. duebeni}, ten tubes with uninfected \emph{G. duebeni}, and five tubes containing no \emph{G. duebeni}, which acted as procedural controls. Fresh mass of \emph{G. duebeni} individuals at the commencement of the experiment were similar across all experimental treatment combinations (ANOVA; $F_7$,$_1$$_3$$_5$ = 1.696,  \emph{p} = 0.115). Turnover rates in the controls were negligible, and did not vary with temperature (Appendix A2; Figure A.2.1.). Two 8 L buckets were kept at each of the four temperatures analysed. The buckets were aerated continually and kept in a 12h:12h light:dark cycle. After 28 days, \emph{G. duebeni} were removed and dissected to ensure infection status. Only organisms with single, cystacanth-stage infections were designated as infected, any hosts with multiple-infections or acanthellae-stage infections were omitted from analyses. Parasites were then examined microscopically to confirm their identity morphologically (following \citet{mcdonald1988}) after cystacanths were first placed in a 0.25 mM solution of sodium taurocholate, a type of bile salt which encourages extension of the proboscis, and left overnight at 37\degree C.

\subsection{Data analyses}

Photographs of the sediment surface of each experimental tube were taken with a Canon EOS 550D (Aperature: f/4.5; Pixels: 5184$\times$3456) and saved as RGB coloured JPEGs. Images were captured under UV light (395 nm wavelength, UV LED Flashlight, LightingEVER, Las Vegas, USA) to optimize fluorophore detection. Images were then processed using ImageJ version 1.43u (US National Institutes of Health, https://imagej.nih.gov/ij/). Images were cropped to appropriate areas using an elliptical cropping shape, then split into red, green and blue colour channels. The red channel was then selected, as it allowed for clearest distinction between the pink fluorophores and the black Lough Lene sediments.  Images were thresholded in order to colour the fluorescent particles white and the sediment particles black. The photo was then analysed and the proportion of black pixels, representing the Lough Lene sand brought up from underneath the fluorophores, recorded. The total area of surface sediment reworked was then quantified in cm$^2$. 

Data were analysed in R version 3.4.1 \citep{r2017}. The extent of sediment surface reworking was log$_1$$_0$-transformed prior to analyses to meet assumptions of normality and homoscedasticity. A linear mixed-effects model was constructed using lme4::lmer \citep{bates2015}, with the log$_1$$_0$-transformed area reworked as the response variable, bucket as a random effect, and temperature and infection status as fixed effects. Model selection with model.sel:MuMIn \citep{barton2018} was conducted and the influence of fixed effects on the model with the lowest AIC is presented.  

\section{Results}

Infected individuals consistently reworked significantly more sediment surface area than uninfected individuals (linear mixed-effects model, F$_1$,$_1$$_3$$_7$ = 7.38, \emph{p} $<$ 0.01;  Fig. \ref{fig:bioturbbox}). Rates of sediment reworking also increased significantly with temperature (linear mixed-effects model, F $_8$,$_1$$_3$$_7$= 5.30, \emph{p}  = 0.05; Fig. \ref{fig:bioturbbox}), though temperature did not interact with parasitic infection in moderating bioturbation. 

\begin{figure}[H]
    \centering
    \includegraphics[keepaspectratio,width=\textwidth]{figures/ch3/bioturbnoctl.pdf}
  \caption [Surface sediment reworked by infected and uninfected gammarids across a broad temperature range.]{Boxplot of the area of sediment turned over by the gammarids at each temperature.} 
    \label{fig:bioturbbox}
\end{figure}


\section{Discussion}

I found that both temperature and parasitic infection impacted the rates at which \emph{G. duebeni} reworked the upper sediment in our experimental microcosms. Moreover, infection and temperature did not interact and moderated bioturbation additively. This comprises the first evidence of which we are aware of parasites enhancing the bioturbation activity of their hosts. Given the importance of gammarid amphipods as key drivers of detritivory and bioturbation in freshwater ecosystems \citep{hunting2012}, coupled with predicted increases in the prevalence of parasites in a warmer world \citep{galaktionov2017}, this finding has important implications for the functioning of freshwater ecosystems under global change.

The enhanced bioturbation caused by infection with \emph{P. minutus} contrasts with our \emph{a priori} predictions. As infection with \emph{P. minutus} increases movement upwards in the water column \citep{jacquin2014, perrot2016, bailly2017}, I expected that infected hosts would interact less with the benthos, leading to a decrease in the rates of surface sediment reworking. However, parasites often do not have fine-scale control when manipulating their hosts. The manipulation of crickets by nematomorph worms provides a clear example. The worms alter the behaviour of crickets to increase their chances of entering the water. However, the manipulation is not a specific push towards the water, but rather results in an increase in erratic jumping \citep{thomas2002}. It has been suggested previously that the behavioural manipulation by \emph{P. minutus} is non-specific and does not driving the intermediate host directly to the exact, preferred definitive host \citep{jacquin2014}. The mechanism of manipulation by the parasite is likely to be similar to that of the closely-related parasite \emph{Pomphorhynchus laevis}, which alters the levels of serotonin in the brain of \emph{Gammarus} in order to control the amphipod and increase its potential for transmission \citep{kaldonski2007, perrot2014}. The increased digging I observed in infected \emph{G. duebeni} may therefore reflect a generalized reduction in risk-aversion behaviour and consequent increase in movement, rather than a mechanism for directly increasing transmission of the parasite to its definitive host. However, further work is needed to determine whether or not enhanced bioturbation activity is adaptive for the parasite. 

A wide range of animal behaviours exhibit thermal dependence, many of which can be explained by metabolic theory \citep{kordas2011, dell2014}. Higher temperatures have been linked previously to enhanced bioturbation rates in non-amphipod aquatic species \citep{ouellette2004}, though the extent to which temperature enhances or supresses bioturbation likely varies across species \citep{maire2010}. My results are consistent with those from previous studies \citep{labaude2016}, and work in this thesis (Chapter \ref{chap:physbeh}) which found additive, and not interactive, effects of temperature and parasitic infection on a range of behaviours in \emph{Gammarus}. As the climate continues to warm, alterations in the prevalence of parasites and associated shifts in the behaviour and functioning of \emph{Gammarus} have the potential to moderate the impact of many of the stressors of aquatic systems associated with global environmental change \citep{baranov2016}. 

Our results demonstrate a significant influence of parasites on the key ecosystem function that is bioturbation \citep{vannatta2018}. Bioturbation has, for example, been linked to the rates of community respiration, sediment transport, nutrient availability \citep{ouellette2004, wohlgemuth2017} and overall community structure \citep{grant1994, croel2011, baranov2016}. Therefore, irrespective of whether or not the altered behaviour we found is adaptive in terms of the parasite’s fitness, our findings have important implications for our understanding of the roles played by parasites in the structure and functioning of entire aquatic systems.