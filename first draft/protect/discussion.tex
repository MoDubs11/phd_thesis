\chapter{Discussion}
\label{chap:discussion}
%\addcontentsline{toc}{chapter}{Discussion}

%---------------------
%
% DISCUSSION
%
%---------------------


\section{The importance of parasites in the functioning of warming ecosystems}

One of the greatest challenges currently facing ecologists is the prediction and mitigation of climate change impacts on the stability and functioning of ecological communities \citep{holling1973, walther2002, donohue2013, nouquet2015, hewitt2016, ullah2018}. The factors that contribute to ecosystem functionality have been hotly debated \citep{grime1997}, though many agree that ecosystem function is likely to be reduced by climate change through reductions in biodiversity \citep{traill2010}. While numerous species face threats from changing climates \citep{thomas2004}, parasites are particularly vulnerable to the impacts of global warming \citep{carlson2017}, and alterations to the relationships between parasites and their hosts are expected as temperatures increase \citep{hoberg2007}. It has been suggested that parasites may play a greater role in ecosystem functioning than other groups \citep{hatcher2012}, due to the critical roles that parasites play in the regulation of nutrient cycling and energy flow \cite{psato2012, buck2017, vannatta2018}, in contributing to biodiversity and community composition \citep{wood2007, grabner2017}, and in the health and behaviour of their hosts \citep{cable2017}. Changes in climate will impact host-parasite interactions, as seen in the interactive effects of temperature and parasites on host respiration (Chapter 2) and the time it took hosts to capture prey (Chapter 4), with potentially significant consequences for overall ecosystem function. 

Recent estimates of ecosystem function have focused on ecosystem multifunctionality, the ability of ecosystems to provide multiple functions and services simultaneously \citep{manning2018}. Though there is disagreement about the definition, measurement, and application of the ecosystem multifunctionality concept \citep{gamfeldt2017, manning2018, meyer2018}, it is likely that multifunctional assessments of ecosystems could provide insight into the fundamental drivers of ecosystem functioning by accounting for the impact of biotic communities on ecosystem-level functional processes \citep{manning2018} and by revealing the impacts of climate change on these processes \citep{traill2010}. Much research has focused on biodiversity as a key driver of the multifunctionality of ecosystems \citep{gamfeldt2017, meyer2018}, though there remains a lack of data and insight into the contribution of individual species to multifunctionality. As parasites have multidimensional impacts on their hosts \citep{thomas2010, cezilly2013}, it is likely that the multiple consequences of parasitic infection have effects on many aspects of host, and potentially ecosystem, function. As shown in Chapters 2 and 3, increasing temperature acts additively with parasitic infection to alter host behaviour, increasing movement upwards in the water column and increasing the rates at which hosts dig into sediments. These findings have significant implications for the contribution of host organisms to ecosystem function, as this increased movement upwards in the water column increases their vulnerability to predation \citep{jacquin2014} with potential consequences for energy flows through ecosystems \citep{buck2017}. Additionally, increased host movement into benthic sediments has important consequences for nutrient flow and oxygenation levels of aquatic ecosystems \citep{baranov2016, wohlgemuth2017}. These multiple impacts of parasites on the multiple functions of hosts within ecosystems suggest likely alterations to ecosystem multifunctionality. Further work, at both the community and whole ecosystem level, to examine the generality of these findings will yield important insights into the role parasites play in maintaining ecosystem functioning in a warming world. 

By thoroughly examining one model system, the acanthocephalan \emph{P. minutus} and its intermediate host, \emph{G. duebeni}, I am able to show that the drivers of host function within ecosystems are diverse and often specific to the function examined, whether it is the additive influence of parasitic infection and temperature on host bioturbation rates (Chapter 3) or the impact of intraspecific competition on host shredding rates (Chapter 4). However, there are limits to the generality of the findings presented in this thesis. The influence of temperature and \emph{P. minutus} on behaviour of the host may vary among different intermediate hosts. For example, behavioural manipulations shown in Chapter 2 with \emph{G. duebeni} as an intermediate host were not found in studies with the intermediate host \emph{Gammarus fossarum} \citep{labaude2017}. Recent work has suggested that parasites typically classified as \emph{P. minutus} may actually be three cryptic species, each of which exhibits some level of host specificity \citep{zittel2018}, making generalisation across host range and across host species considerably more difficult. However, functionally similar behavioural manipulations have been shown to arise from different mechanisms, as similar changes in geotaxis and phototaxis of host amphipods have been linked to different mechanisms among different parasite species, with \emph{P. minutus} likely causing changes by increasing anaerobic metabolites \citep{perrot2016} and \emph{Pomphorhynchus laevis} altering serotonin concentrations \citep{perrot2014}. While the mechanism of manipulation may differ among parasite species, the consequences for ecosystem functioning may be identical. Additionally, due to the fact that \emph{P. minutus} utilizes avian definitive hosts, I was unable to either experimentally infect intermediate hosts or determine the consequences of infection on the function of the definitive host within these ecosystems. Finally, amphipods may not function identically across their native and invasive ranges, which has consequences for the generality of the findings described in this thesis \citep{little2018, tonin2018}. These limitations, however, are somewhat counterbalanced by the pervasiveness of \emph{P. minutus} and \emph{G. duebeni} in Irish ecosystems (Chapter 1) and the nearly global distribution of amphipods and acanthocephalans.

Throughout this thesis, I highlight the joint impacts of temperature and parasites on their hosts within the context of the ecosystems in which they are found, as each impacts specific components of host function. By integrating studies from the laboratory and the field into a single framework, we show that findings from the laboratory scale to the field and that \emph{in situ} experimental systems may provide the most accurate estimates of the influence of temperature and parasites on host function. Focusing only on trophic or non-trophic components of interactions between hosts and parasites likely underestimates their impact on ecosystem function \citep{buck2018}. The data presented here emphasize the need for a holistic approach to ecosystem-level ecology that is built upon a strong understanding of the populations that persist and the biotic interactions within ecosystems. 

\section{Future prospects} 

\subsection{Community Dynamics}

Ecological communities are complex systems and the impacts of global climate change at the community level are likely to be both significant and difficult to predict \citep{kordas2011, hewitt2016}. Prior work has shown that variability in a single species can have impacts on whole community responses to disturbance \citep{mcclean2015, mrowicki2016}, suggesting that the joint effects of parasitism and warming on host populations described in this thesis may have broader consequences for community dynamics. As parasite ecology is increasingly informed by community ecology to predict the impacts of parasites outside of their host organisms \citep{johnson2015}, the need for empirical data focused on the community-level impacts of parasites grows exponentially \citep{poulin2018best}. 

\emph{Polymorphus minutus} has a multi-host lifecycle, cycling between eggs in the environment, larvae in amphipods, and adult stages in water fowl. As such, \emph{P. minutus} parasites have impacts on hosts across multiple trophic levels within the biological community, as well as potentially altering the competitive balances between species within the same guild \citep{macneil2003}. Multiple hosts and parasites interact to contribute to ecosystem multifunctionality, increasing the potential for the impacts of parasites on ecologically important hosts to be either magnified or reduced. The impact of parasites on host function can be altered by host-community dynamics. For example, the influence of parasites on the feeding ecology of predatory hosts can be altered by the presence of predators \citep{reisinger2016} and competitors \citep{paterson2014}. These community-level dynamics, particularly interspecific competition, have been shown to moderate the impact of parasites on their hosts and their ecosystems with consequences for ecosystem function \citep{kordas2011}. Moreover, parasites also form communities within hosts, and studies on the parasite communities within hosts have shown influences of within-host parasite community dynamics on host functioning \citep{dezfuli2001, hafer2016, kirk2018}. Mesocosm studies on the influence of warming and parasitic infection on community level dynamics are needed to help elucidate the relative importance of both temperature and parasites beyond their effects on host populations \citep{woodward2010}. As it is comparably easy to alter overall numbers of parasites within amphipod communities while setting up mesocosms, future work with different levels of parasite prevalence within realistic communities could yield important insight into the impacts of parasitism on the function of hosts under intraspecific and interspecific competition. 

Perhaps most importantly, parasites with complex life cycles, such as the acanthocephalans studied in this thesis, are likely to have impacts beyond those on their intermediate hosts \citep{vancleave1951, moenickes2011}. Alterations in temperature, both in terms of absolute temperature and temperature variability, may impact transmission rates, definitive host parasite burden, and the survival of any eggs within the environment \citep{lenihan1999, poulin2006, gehman2018}. Community-level studies that include all hosts and allow for multiple transmission cycles, though logistically difficult, could shed light on the broader influence of parasites in their ecosystems beyond a direct functional approach.

\subsection{Ecosystem stability }

As parasites and temperature are important drivers of overall ecosystem function, they likely play a major role in moderating the stability of ecosystems \citep{lafferty2008, jephcott}. Ecosystem stability is a complex concept, though recent work has begun to recognize its multidimensional nature and unite its varied components into a logical framework \citep{pimm1984, donohue2013, don2016ohu, hillebrand2018}. Even with newly developed analytical and computational models, the specific role of parasites in overall ecosystem stability is difficult to elucidate and has been explored solely in a trophic context \citep{dunne2013}. In theoretical models, host-parasite interactions are typically added directly to food webs, ignoring the differences between host-parasite and predator-prey relationships, though recent work has aimed to more comprehensively examine the impacts of parasites in a generalized consumer-resource framework \citep{lafferty2015}. The incorporation of parasitic links into food webs has been shown to both increase and reduce the stability of the network \citep{wood2015}, highlighting the lack of general consensus on the impacts of parasites and the potential for additional empirical research to clarify the drivers of these stabilizing or destabilizing relationships. As shown in this thesis and in the literature \citep{perrot2014}, the impacts of parasites on their hosts are often multidimensional themselves. Future studies on the relative impacts of parasitic links on network connectance, network stability, and the strength of inter-specific interactions should consider the non-trophic, as well as the trophic impacts of parasites in their ecosystems. By comprehensively examining the role parasites play in the multiple components of ecosystem stability and coupling these results with the known impacts of parasites on ecosystem function, we will be able to develop more accurate predictions on the potential impacts of climate change on ecosystems.

\section{Concluding Remarks}

Throughout this thesis, I demonstrate the critical importance of parasites and temperature for the function of hosts and their populations. By integrating laboratory experiments and \emph{in situ} studies, the specific and interactive drivers of host-parasite interactions become apparent. The generality of my conclusions based on the work presented in this thesis could be further elucidated through additional experimental work within multiple host-parasite systems, particularly in systems with functionally important microparasites, such as \emph{Daphnia spp.} \citep{johnson2006, kirk2018}. While theoretical investigations have provided excellent insights into the influence of temperature on the relationships between parasites and their hosts \citep{barber2016, cohen2017, gehman2018}, we require robust empirical evidence about the influence of temperature and parasites on host function before practical solutions for the maintenance of ecosystem function through changes in climate may be proposed \citep{rohr2011, altizer2013}. A mechanistic understanding of the potential impacts of temperature, including the work presented in this thesis, can be combined with predictive modelling and strong theoretical understanding of ecosystem-level ecology to manage and protect the parasites of the future with the aim of enhancing ecosystem health and stability. 


%\bibliography{References}