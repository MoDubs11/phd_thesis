\chapter[Individual and combined effects of warming, intraspecific competition and parasitic infection on detritus processing rates \emph{in situ}.]{Individual and combined effects of warming, intraspecific competition and parasitic infection on detritus processing rates \emph{in situ}.}
\label{chap:shannon}

\section{summary}
One of the greatest challenges facing ecologists is how to detect and predict the impact of parasites on their hosts in warming ecosystems, as hosts face biotic pressures. Here, I examine the individual and combined effects of warming, intraspecific competition and parasitic infection on feeding behaviour of an important aquatic ectotherm, \emph{Gammarus duebeni}, in the field. I found that \emph{per capita} rates of detritus processing by \emph{G. duebeni} decreased with population density. In addition, high \emph{per capita} processing rates at low densities enabled a few individuals to process as much leaf litter as groups of up to ten-fold greater densities. Further, both moderate warming (2\degree C) and infection with acanthocephalan parasites increased survival rates of \emph{G. duebeni}. The increased survival of infected hosts \emph{in situ} should be considered in future attempts to analyse the transmission rates of larval acanthocephalans, particularly as the behavioural changes associated with these parasites may counterbalance the increased survival. Overall, these results demonstrate the important, and often underappreciated, roles of intraspecific competition and parasitic infection in moderating ecosystem functioning and highlight the need for their inclusion in models predicting the consequences of global environmental change.

\section{Introduction}

Global climate change is expected to have significant impacts on the structure and function of aquatic ecosystems \citep{carpenter2011, ogorman2012}. As the climate warms, the nature and strength of biotic interactions are likely to change. Intraspecific competition is a particularly important determinant of  the overall functioning and stability of ecosystems \citep{barabas2016}, and of the persistence of species in variable climates \citep{pilfold2014, stenseth2015}. Though there is some theoretical understanding of how temperature could moderate intraspecific competition \citep{amarasekare2015}, empirical evidence is lacking. Metabolic scaling theory suggests that greater activity levels at higher temperatures should increase \emph{per capita} acquisition of resources, and thus reduce carrying capacity \citep{savage2004}. Alternatively, intraspecific competition may be greatest at the thermal optimum for reproduction due to a peak in demand for resources. Empirical quantification of the individual and combined influence of intraspecific competition and temperature on the species' functioning is needed improve the predictive power of theoretical models and support more robust management of vulnerable ecosystems. 

Parasites play keystone roles in ecosystems \citep{hatcher2008}, and influence the intensity of both intra- and interspecific competition \citep{yan1995, macneil2003}. The close coevolution of parasites and their hosts places parasites at particular risk to stress from climate change \citep{carlson2017}. Increasing temperatures have, therefore, the potential to decouple host-parasite relationships \citep{goedknegt2015, strepparava2017}. However, the influence of climate warming on host-parasite relationships under differential levels of intraspecific competition has not yet been examined. 

Here, I examine whether warming and parasitic infection interact to moderate the density dependence of a key ecosystem function in aquatic ecosystems, the rate of detritivory. I tested this in the field using the model host-parasite system of \emph{Gammarus duebeni} and its acanthocephalan parasite \emph{Polymorphus minutus}. I used thermal effluents from a power plant to provide \emph{in situ} warming. Though many detritivorous species contribute to overall leaf litter breakdown in aquatic ecosystems \citep{tonin2018}, amphipods in particular dominate many benthic macroinvertebrate assemblages and contribute significantly to detritus processing \citep{little2018}. \emph{Gammarus} species are ectothermic amphipods that are widespread geographically and account for up to three quarters of leaf litter degradation in European freshwater systems \citep{piscart2011}. \emph{Gammarus duebeni} var. \emph{celticus} is the most abundant amphipod in Irish lakes and rivers \citep{reid1938}, acting as an omnivorous shredder and a valuable food source to higher trophic levels \citep{kelly2002}. Gammarid amphipods are a particularly suitable model for the questions we seek to address, as infection with parasites has been shown to alter the feeding rates of gammarid species in laboratory experiments \citep{bunke2015, labaude2016, laverty2017, Chapter 4}. Moreover, the larval cystacanth of \emph{Polymorphus minutus} is found frequently in amphipods in many Irish waters, and this infective stage reduces the fecundity of its hosts and alters their behaviour \citep{bailly2017}. 

Thermal effluents are an often overlooked source of increased temperatures for warming experiments in natural communities in the field. Power plants often intake water from sources near their location and dispel warmed coolant water into local water systems, where the alteration in temperature creates microclimates within larger water bodies \citep{aho1982, hoglund1990}, providing realistic complex systems for the study of climate warming effects \citep{raptis2016}. 

I hypothesized that warming would cancel out any reductions in shredding rates caused by infection \citep{labaude2016}, and group size would significantly reduce per capita shredding rates due to increased intra-specific competition at high density \citep{vandervorste2017}.

\section{Method}

\subsection{Study site}

The Lough Ree power station in Lanesborough, Co. Longford (53\degree 40'29.8"N 7\degree 59'30.3"W), has been discharging warmed effluent water into the River Shannon since 1958. The thermal effluent is raised up to 10\degree C above ambient conditions. We utilized two 100 m reaches (\ref{fig:shannonmap}) within the River Shannon as experimental locations, with the thermal effluent entering one reach but not the other. The two reaches are separated by a barrier island that prevents water mixing. The reaches are ideal for temperature-based experiments, as the warmed (effluent) reach has similar physical and chemical characteristics to those in the main river. 

\begin{figure}%[H]
    \centering
    \includegraphics[scale=0.9]{C:/Users/Administrator/Google Drive/PhD General/Writing/THESIS/5_Shannon/shannonmap.png}
  \caption [Map of experimental site in the River Shannon]{The location of the warmed (red) and ambient (blue) reaches used in our experiments. The Lough Ree power station thermal discharge is indicated with a yellow star.} 
    \label{fig:shannonmap}
\end{figure}

\subsection{Experiments}

I used Onset UA002 HOBO temperature loggers to track water temperature every 10 minutes within the experimental reaches. Because river flows modified the temperature differential between the warmed and ambient reaches (\ref{fig:shannontemp}a), I ran my experiment twice: once during high flows in October 2017 (Winter Experiment) and then again during low flows in June 2018 (Summer Experiment). In winter, high water levels, with flow rates in the top 5\% of flow rates measured annually, diluted the thermal effluent more than anticipated \citet{opw2018}. The high flows reduced the temperature differential between the heated and ambient reaches, though a temperature differential of between 0 and 2.4\degree C was recorded (\ref{fig:shannontemp}ab). During summer, high air temperatures and very low rainfall generated high ambient water temperatures (\ref{fig:shannontemp}a). As the warmed treatment is heated above ambient temperatures, the heated treatment reached temperatures in excess of 30\degree C in summer, with the temperature differential between ambient and heated reaches exceeding 8\degree C (\ref{fig:shannontemp}b). 

\begin{figure}%[H]
    \centering
    \includegraphics[scale=0.9]{C:/Users/Administrator/Google Drive/PhD General/Writing/THESIS/5_Shannon/shannontemp.png}
  \caption [Temperature of the experimental sites in the River Shannon]{Temperature (a) and temperature differential (b) of warmed and ambient reaches during the two experimental runs. Day refers to day number within the experimental run.} 
    \label{fig:shannontemp}
\end{figure}

Amphipods used in the experiments were collected from Lough Lene (53\degree 39'37.6"N 7\degree 11'41.7"W) in October 2017 and June 2018. Only adult amphipods were used in experimental trials. I quantified rates of \emph{Gammarus} detritivory using leaf litter decomposition bags \citep{benfield2006}. These were constructed using double-layered mesh (20 cm$x$30 cm, aperture 1 mm), each containing 7 g of dried horse chestnut leaves, \emph{Aesculus hippocastanum}, \citep{agatz2014}. Background rates of microbial decomposition were quantified using 0.5 g of dried horse chestnut leaves placed within a smaller double-layered, sealed mesh bag that was inaccessible to amphipods. Litter bags were conditioned in aerated lake water for 24 hours to increase palatability \citep{graca1993, agatz2014}, after which amphipods were added to the larger bags in numbers appropriate to experimental treatments. There were three levels of amphipod densities: 10 individuals (low density); 25 individuals (intermediate density) and 100 individuals (high density). As there were insufficient numbers of infected gammerids to explore the effects of parasitic infection at the highest level of \emph{Gammarus} density, parasite infection was crossed with just two levels of amphipod density (i.e.\ low and intermediate density). Each experimental treatment combination was replicated five times.

Decomposition bags were placed in the river attached to cinderblocks to maintain their position at the experimental site. The bags were deployed over four weeks in winter (16 November 2018 - 14 December 2018) and three weeks in summer (4 July 2018 - 25 July 2018). I shortened the duration of the Summer Experiment due to high rates of mortality, which resulted in the death of all amphipods in the experimentally warmed reach. At the end of the experiments, bags were returned to the laboratory and leaf litter, gammarid amphipods, and control bags were separated from each other. Leaves were dried at 60\degree C for 48 hours and weighed. All \emph{G. duebeni} individuals used in the experiment were dissected to confirm infection status. Larval cystacanths were dissected out of hosts and placed in a 0.25 mM solution of sodium taurocholate. Cystacanths were incubated in the sodium taurocholate at 37\degree C for 12 hours to encourage the extrusion of the hooked proboscis. Cystacanths were examined microscopically to confirm their identity, following \citet{mcdonald1988}.

The absolute mass of leaf litter decomposed was calculated as the difference between the dry mass of leaves before and after the experiment. This was then standardized to account for microbial degradation and leaching, the two major sources of leaf litter loss not due to shredding \citep{cummins1979}, using the small bag controls which were included within each replicate bag. A microbial degradation factor was determined for each bag as the difference between the dry weights of the small bag leaves before and after the experiment as a proportion of the initial dry weight of the small bag leaves. The corrected amount of leaves shredded was the change in leaf weight minus the product of the initial leaf weight and the microbial degradation factor \citep{benfield2006}. The corrected shredding amount was then divided by the mean number of \emph{G. duebeni} in the bag over time and the number of days the bag was immersed \emph{in situ}. I therefore quantified three main response variables: the shredding rate per individual amphipod per day (after accounting for microbial degradation), the absolute shredding rate (after accounting for microbial degradation), and the survival rate of amphipods.

\subsection{ Data analyses}

All data analyses were done using R \citep{r2017}. I used linear models to determine how temperature, parasitic infection, and intraspecific competition separately and collectively influence the overall shredding and survival rates of gammarids, while the \emph{per capita} shredding rate was analysed using a generalized linear model with a Gamma distribution and an inverse link function to account for non-normality of the data. Given the full mortality of gammarids in the warmed river reach, data from the summer experiment were omitted from these analyses. In addition, I explored whether the individual and combined effects of density dependence and parasite infection on \emph{Gammarus} detritivory and survival varied over the year. This was done using linear models for both the total amount shredded and survival and a generalized linear model with an inverse Gaussian distribution for \emph{per capita} detritivory. Season (i.e.\ summer or winter) was incorporated as a fixed effect in these analyses. 


\section{Results}

Even though the total amount of leaf litter processed did not vary with the density of \emph{G. duebeni}, warming, or infection (\ref{tab:winterstats}, \ref{fig:wintershred}a), \emph{per capita} shredding rates varied significantly with \emph{Gammarus} density (\ref{tab:winterstats}), with the highest rates of individual shredding in lowest density groups (\ref{fig:wintershred}b). Neither infection with \emph{P. minutus} nor warming influenced shredding rates, and there were no interactions between group size, warming, and infection status, indicating that warming and infection did not modify the density-dependence of feeding rates (\ref{tab:winterstats}, \ref{fig:wintershred}b).  

\begin{figure}%[H]
    \centering
    \includegraphics[scale=0.9]{C:/Users/Administrator/Google Drive/PhD General/Writing/THESIS/5_Shannon/wintershred.png}
  \caption [Shredding by \emph{G. duebeni} at ambient and warmed temperatures in the field.]{The total amount shredded (a) and \emph{per capita} shredding rates (b) at each \emph{Gammarus} density (10, 25 and 100 individuals) are shown for infected and uninfected individuals.} 
    \label{fig:wintershred}
\end{figure}

Both infection with \emph{P. minutus} and warming increased survival rates of \emph{Gammarus} significantly, though group size had no effect (\ref{tab:winterstats}, \ref{fig:wintersurvive}).

\begin{figure}%[H]
    \centering
    \includegraphics[scale=0.9]{C:/Users/Administrator/Google Drive/PhD General/Writing/THESIS/5_Shannon/wintersurvive.png}
  \caption [Survival of \emph{G. duebeni} in ambient and warmed reaches in the field.]{Proportion of infected and uninfected \emph{Gammarus duebeni} surviving in ambient and heated reaches at low (10 individuals) and intermediate (25 individuals) densities.} 
    \label{fig:wintersurvive}
\end{figure}

\begin{table}[]
\caption[Individual and combined effects of group size, parasite infection and warming on rates of detritivory and survival of \emph{Gammarus duebeni}. Statistically significant effects are shown in bold.]{Results of statistical analyses for winter field experiments.}
\label{tab:winterstats}
\begin{tabular}{lllllllllllllll}
Response variable & \multicolumn{2}{l}{Density} & \multicolumn{2}{l}{Infection} & \multicolumn{2}{l}{Warming} & \multicolumn{2}{l}{Density*Infection} & \multicolumn{2}{l}{Infection*Warming} & \multicolumn{2}{l}{Warming*Density} & \multicolumn{2}{l}{Warming*Infection*Density} \\
\textit{} & \textit{F} & \textit{p} & \textit{F} & \textit{p} & \textit{F} & \textit{p} & \textit{F} & \textit{p} & \textit{F} & \textit{p} & \textit{F} & \textit{p} & \textit{F} & \textit{p} \\
Daily shredding per capita & \textbf{64.19} & \textbf{0.00} & 2.32 & 0.14 & 0.16 & 0.69 & 2.55 & 0.12 & 0.81 & 0.37 & 0.25 & 0.78 & 0.43 & 0.52 \\
Shredding total amount & 3.86 & 0.06 & 1.97 & 0.15 & 0.66 & 0.42 & 1.05 & 0.31 & 1.11 & 0.30 & 0.81 & 0.45 & 0.30 & 0.59 \\
Survival & 2.07 & 0.16 &  \textbf{5.11} & \textbf{0.03} & \textbf{10.26} & \textbf{0.00} & 0.05 & 0.82 & 0.04 & 0.84 & 3.92 & 0.06 & 0.30 & 0.59
\end{tabular}
\end{table}

The density dependence of \emph{per capita} shredding rates varied significantly between the winter and summer experiments (\ref{tab:seasonstats}), with a significantly stronger effect of \emph{Gammarus} densities in winter (\ref{fig:seasonshred}). This suggests that the key role of intraspecific competition on gammarid shredding rates varies over the year. 

\begin{table}[]
\caption[Results of analysis of interannual variation in rates of shredding and survival of \emph{Gammarus duebeni}. Statistically significant effects are shown in bold.]{Results of statistical analyses for ambient reaches in the field.}
\label{tab:seasonstats}
\begin{tabular}{lllllllllllllll}
Response variable & \multicolumn{2}{l}{Density} & \multicolumn{2}{l}{Season} & \multicolumn{2}{l}{Infection} & \multicolumn{2}{l}{Density*Infection} & \multicolumn{2}{l}{Infection*Season} & \multicolumn{2}{l}{Season*Density} & \multicolumn{2}{l}{Season*Infection*Density} \\
\textit{} & \textit{F} & \textit{p} & \textit{F} & \textit{p} & \textit{F} & \textit{p} & \textit{F} & \textit{p} & \textit{F} & \textit{p} & \textit{F} & \textit{p} & \textit{F} & \textit{p} \\
Daily shredding per capita & \textbf{30.53} & \textbf{<0.01} & 0.28 & 0.6 & 2.36 & 0.13 & 0.82 & 0.37 & 0.58 & 0.45 & \textbf{2.79} & \textbf{0.01} & 1.89 & 0.18 \\
Shredding total amount & \textbf{8.23} & \textbf{0.001} & 0.23 & 0.63 & \textbf{4.88} & \textbf{0.03} & 0.48 & 0.49 & 0.45 & 0.51 & 1.98 & 0.15 & 0.95 & 0.34 \\
Survival & 2.46 & 0.13 & 2.15 & 0.15 & \textbf{6.56} & \textbf{0.02} & 0.62 & 0.44 & 0.36 & 0.55 & 2.56 & 0.12 & 1.14 & 0.29
\end{tabular}
\end{table}

\begin{figure}%[H]
    \centering
    \includegraphics[scale=0.9]{C:/Users/Administrator/Google Drive/PhD General/Writing/THESIS/5_Shannon/seasonshred.png}
  \caption [Shredding by \emph{G. duebeni} during the winter and summer field experiments.]{The total amount shredded (a) and \emph{per capita} shredding rates (b) at each Gammarus density (10, 25, and, 100 amphipods) are shown for infected and uninfected individuals.} 
    \label{fig:seasonshred}
\end{figure}

\section{Discussion}

My results demonstrate the importance of intraspecific competition in moderating a key ecosystem function, rates of detritivory, in aquatic ecosystems. Moreover, these effects also appear to vary seasonally, being stronger in winter than in summer. Intraspecific competition for resources is a key driver of amphipod feeding behaviour even when food resources are relatively abundant, a finding which is in concurrence with laboratory studies on amphipod feeding behaviour \citep{mancinelli2012, labaude2016, vandervorste2017}. The sex, body size, mouthpart shape, physiology, and behaviour of amphipods have all been found to moderate shredding rates in the laboratory \citep{rota2018}. Our findings indicate that the density-dependence of shredding rates should be considered in future estimates of amphipod leaf litter shredding. Given that allochthonous detritus contributes significantly to the overall carbon available in aquatic ecosystems \citep{wallace1997}, these findings have important implications for energy transfer and food-web stability in aquatic ecosystems in general.

I found no effect of warming on rates of detritivory. Though this was an unexpected finding in light of previous research \citep{labaude2016, pellan2016, laverty2017, Chapter 4}, it may be a consequence of the relatively small temperature differential observed between the ambient and warmed reaches in winter. Nonetheless, temperatures in the warmed reach were approximately 2\degree C higher than ambient conditions over at least fifteen days towards the end of the experimental period. Given the results of my previous experiments, particularly those described in Chapters 2 and 4, such a temperature differential would have been expected to have caused some detectable shifts in gammarid feeding behaviour. Unfortunately, the attempt to examine the influence of warming in summer was confounded by exceedingly high temperatures in the heated reach which caused mortality of all experimental animals. I recommend that future work in this system should focus on experiments in the autumn and spring, when ambient temperatures are generally mild and temperature differentials will be higher than in winter. 

High \emph{per capita} shredding rates of \emph{Gammarus} at low densities in winter enabled a few (10) individuals to process as much leaf litter as groups of up to ten-fold greater densities. Given the importance of shredders in low order aquatic ecosystems \citep{vannote1980}, and the significant role of shredders in leaf litter processing globally \citep{petersen1974}, these findings have important implications for overall ecosystem function and nutrient flow. Moreover, infection with \emph{P. minutus} reduced the \emph{per capita} shredding of amphipods when examined across seasons. Laboratory examination of the impact of acanthocephalan parasitism on the shredding rates of gammarid amphipods has found that infection either reduces \citep{agatz2014, labaude2016} or does not affect shredding rates \citep{fielding2003, Chapter 2}. These disparities are perhaps due to the varied temperatures at which these rates are measured and potential influences of the multiple source populations of hosts and parasites examined. The impact of acanthocephalans on the feeding behaviour of their hosts may also be species-specific and context-dependent. 

I found that infection with \emph{P. minutus} parasites increased survival of \emph{G. duebeni} significantly. Some laboratory experiments have found higher mortality rates of amphipod hosts when infected with larval acanthocephalans \citep{brown1989, labaude2015influence}, while others have found no effect on survival \citep{chen2015}, or that acanthocephalan infection even confers some advantages to hosts, such as increased salinity tolerance \citep{piscart2007}. Survival of the intermediate host is beneficial to \emph{P. minutus}, which relies on the amphipod’s survival for transmission, and it seems a complex and realistic environment provides the context in which parasitic infection confers a survival advantage.  

Warming enhanced gammarid survival in winter, though the high levels of warming in summer caused 100\% mortality in our experimental populations, likely due to increased metabolic demand \citep{galic2017}. Temperature within the warmed reach in winter was well within the annual range experienced by amphipods in their native range, which may explain the higher survival levels, as ambient temperatures at the time were close to their annual minimum. 

As temperatures continue to increase, it is crucial for future management of economically and ecologically valuable ecosystems to understand how multiple stressors interact in realistic field studies \citep{ogorman2014, jackson2016}. My results show that intraspecific competition was the key driver of shredding behaviour of ecologically important detritivores, a finding with implications for ecosystem functioning and for any future predictions of ecosystem structure and function. 
